\begin{appendix}
    \section{Appendix}
    Reproducibility, the ability to reproduce computational results using identical data and software, 
    is a cornerstone of the scientific methodology. However, through the past decade, several studies 
    revealed a widespread lack of results’ reproducibility, to the point that the existence of a 
    reproducibility crisis is now acknowledged in various fields.
    In Machine Learning, given the flexibility available in various phases of constructing a 
    computational model, the experiments are not immune to reproducibility issues either. 
    In case of imbalance learning for problems with multiple classes, the problem is even more 
    severe since there are more parameters in play for constructing a model. 
    The resulting reproducibility challenges have implications in various disciplines 
    including bioinformatics, the primary focus of our study.
    
    
    Researchers have already taken counter-measures proposing various recommendations for having results’ 
    reproducibility in this domain of study. Some conferences (e.g. NeurIPS) have even adopted 
    new measures in that regard. Following those guidelines could ensure reproducibility  
    to an agreeable degree in balanced problems.
    In this work we demonstrate that in an imbalanced scenario, even in its basic form, 
    a study report with a fair amount of details, could reproduce a wide range of results if 
    methodological flexibility is permitted. The flexibilities that may not affect the results 
    much in a generic balanced scenario. 
\end{appendix}