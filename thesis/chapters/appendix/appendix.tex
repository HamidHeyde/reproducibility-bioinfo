\chapter{Reproducible Experiment Report}
\label{appendix:a}

\begin{table}[ht]
    \scriptsize
    \begin{adjustwidth}{-0.0in}{-0.0in} 
        \centering
        \begin{tabular}{|l|}
            
            \hline
            \multicolumn{1}{|g|} {Data Provenance and Sharing}\\
            \hline
                \textbullet {Report on the source(s) of the raw data}\\
                \textbullet{Explain the curation process}\\
                \textbullet{Share the final dataset.}\\
            \hline
            \multicolumn{1}{|g|} {Feature Provenance and Sharing}\\
            \hline
                \textbullet{Explain the concept associated with the extracted feature }\\
                \textbullet{Explain the process through which the feature is extracted}\\
                \textbullet{For formulas, describe the associated parameters}\\
                \textbullet Share the extracted feature file (or reasonable amount of 
                the final extracted feature file)\\
            \hline
            \multicolumn{1}{|g|} {Model Provenance and Sharing:}\\
            \hline
            \multicolumn{1}{|c|} {Data Pre-processing}\\
            \hline
                \textbullet{Explain the pre-processing technique concept along with any involved process, formula and parameters}\\
                \textbullet Share the final transformed feature file (or reasonable amount of 
                the final transformed feature file)\\
            \hline
            \multicolumn{1}{|c|} {Model Structure}\\
            \hline
                \textbullet{Explain any applied sampling technique along with the process formula and parameters }\\
                \textbullet{Describe the strategy for splitting the original data into train, validation and test}\\
                \textbullet{For problems with multiple classes, describe the decomposition strategy}\\
                \textbullet Explain the deployed algorithm, considered range of hyper-parameters and the associated values for 
                obtaining \\the published results. For multiple classes, this process should be done for all the 
                decomposed models.\\
                \textbullet If a specific optimal hyper-parameters search technique is used, provide the final deployed values 
                resulted\\ from the process, describe the method, any involved parameter(s), the process and 
                how it has been \\ applied to the model(s).\\
                \textbullet For problems with multiple classes (or ensemble models), report on the structure, describe the 
                underlying\\ models and the aggregation strategy and how all those were applied to the model to generate 
                the final\\ results. If the results are generated using a different threshold rather 
                than using\\ the default one used by the classification algorithm, report on the threshold value for each 
                decomposed model.\\
                \textbullet{Share reasonable amount of the generated results (where possible)}\\
            \hline
            \multicolumn{1}{|c|} {Model Evaluation}\\
            \hline
                \textbullet Describe the choice of statistical method used for evaluation of the results, any involved formula 
                \\ and its parameter(s)\\
                \textbullet{If averaging through multiple results, describe the technique (micro vs macro)}\\
                \textbullet{Define error bars (if any)}\\
            \hline

        \end{tabular}
    \end{adjustwidth}    
\end{table}