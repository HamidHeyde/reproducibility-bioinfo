\section {Conclusion}

\subsection{Software Provenance and Sharing}
    \label{sec:softwareProvenance}

    From the data curation and putting together a dataset, to model training and evaluation, different software and coded programs 
    could get involved in some phase of a typical bio-informatics related problem. 
    Throughout the replication process, failing to use the same software (along with its parameters 
    being set to the same state or value at the time of the experiment) through the same environment, can also affect the final results. 
    The report on this aspect should cover all the software, coded programs and libraries details being involved throughout the whole process. 
    If the pipeline \footnote{In software engineering, a pipeline consists of a chain of processing elements, 
    arranged so that the output of each element is the input of the next[wikipedia]} is environment-dependent, then,  It should also include the environment-related 
    details in which the pipeline has been executed.\\

    So, if any software has been involved in any part of the process, software-related details (version, dependencies, etc.) along 
    with the environment-related details (operating system name, version, etc.), should be included in the report.
    Also, in case of a coded program and any involved library, related details (programming language, version, dependencies, etc.) 
    should be included in the report. Sharing the coded program through available means (i.e. GitHub) is considered a better practice.\\

    Another important point to note is to version control the whole pipeline. Changing direction in an experiment or setting new targets 
    for the study (while working on the same problem) is a common phenomenon that can take place throughout any research. 
    In that case, you should make sure to label the original work, and start the new work with a copy of the original pipeline. If your 
    current work is based on any other work, make sure to include all the details in your report or provide a link to the correspondent 
    resource (paper, website, repository, etc.) that covers all the related details. Using a version control system (i.e. GitHub) 
    that records changes to a file or set of files over time is a good practice as it would allow recall on specific versions 
    later if needed.\\
    

    Another solution to this problem is the use of containers \footnote{OS-level virtualization refers to an operating system paradigm 
    in which the kernel allows the existence of multiple isolated user space instances. Such instances, called containers, Zones, 
    virtual private servers, partitions, virtual environments, virtual kernel or jails, may look like real computers from 
    the point of view of programs running in them.[wikipedia]} (i.e. Docker) or virtual machines \footnote{In computing, 
    a virtual machine (VM) is an emulation of a computer system. Virtual machines are based on computer architectures and 
    provide functionality of a physical computer.[wikipedia]} (i.e. VMware Workstation) through which researchers could create  
    a virtual computer and set up the experiment pipeline on that. They could then, leave the experiment 
    pipeline as it is (including all the involved software, coded programs and dependencies along with the environment) 
    in the state that the results are produced. 
    The experiment could then be shared through sharing the virtual environment. So, any 
    further research on the same problem could be conducted by loading the container (or the virtual machine) and running 
    the same experiment through the same environment. This would significantly speed up the process as there would be no need for  
    replication process.

\myparagraph{Reproducible Experiment Report}
\begin{table}[ht]
    \centering
    \begin{tabular}{| P{12cm} || p{4cm} |}
        \hline
        \rowcolor{gray}\multicolumn{2}{|L|}{Data Provenance and Sharing} \\
        \hline \hline
        Original Results & 74.65 \\
        \hline \hline
        \rowcolor{gray}\multicolumn{2}{|L|}{Feature Provenance and Sharing}\\
        \hline \hline
        \rowcolor{gray}\multicolumn{2}{|L|}{Model Provenance and Sharing}\\
        \hline \hline
        \rowcolor{gray}\multicolumn{2}{|L|}{Software Provenance and Sharing}\\
        \hline \hline
        
        $\bullet$
        {Create a Jupyter Notebook to walk the reader through the pipeline starting from the feature extraction 
        all the way to result generation. for each step:}&\\ 
        \quad $\bullet$ 
        {Explain briefly what the code does (or make a reference to the correspondent part in the paper) 
        followed by runnable code that demonstrates how to run the code}&\\ 
        \quad $\bullet$ 
        {if running the whole code is not possible, you can provide explanation over how to run the code and provide the result
        on the page as a reference on what to expect to make sure that you have run the code correctly}&\\ 
        \quad $\bullet$ 
        {make the code modular and show the reader how to run tha module and provide proper arguments }&\\ 
        $\bullet$ 
        {host the project on github or any alternative mean which provides version controlling. so, the required changes could be traced}&\\ 
        $\bullet$ 
        {Environment: if the pipeline is dependent on any specific environment, provide details over the environment details}&\\ 
        $\bullet$ 
        {if you have used a third party software or a provide details and link to the documentation}&\\ 
        $\bullet$ 
        {coded programs: provide languages and dependencies}&\\ 

        \hline

    \end{tabular}
    \captionsetup{font=small,width=12cm}
    \caption{Reproducible experiment checklist for a problem with imbalanced dataset}
    \label{tab:table3}
    
\end{table}