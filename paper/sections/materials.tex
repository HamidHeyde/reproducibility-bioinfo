\section{Materials and Methods}
    ==> Normally, in machine learning related problems, the solution is divided into 4 phases, 1. putting 
        together a dataset, 2. data preprocessing, 3. model building and 4. model evaluation\\
        Some discussion over how the whole process works.(here or in introduction)\\

    \subsection{Dataset}

    The dataset contains 780 transporter proteins classified into 7 substrate specific classes (70 amino acid transporters, 
    60 anion transporters, 260 cation transporters, 60 electron transporters, 70 protein/mRNA transporters, 60 sugar 
    transporters, 200 other transporters) and one none-transporter class (600 proteins) for a total of 1,380 protein 
    sequences which are available at  \trssp{?dowhat=Datasets}{TrSSP} website.
    
    For the purpose of our study, we coded a program (\git{download.py}{download.py}) to download all the sequences from the 
    \ncbi{protein}{NCBI} database using the sequence accession number from TrSSP website. 
    Considering the fact that some sequences could be updated through time, we checked all the \git{dataset/trainTest/}{downloaded} 
    sequences against the originals to make sure all of our sequences match those of the TrSSP. We then updated the 
    modified sequences and their accession numbers accordingly.

    \subsection{Feature extraction}

    The authors \cite{mishra2014prediction} have extracted five features out of the dataset sequences. They have then built-up different Support Vector Machine based 
    computational models using a combination of these features. Following the descriptions from the paper, we generated all the 
    \git{features/}{features} in Comma Separated Values (.csv) format adding one more column for the labels.

    Table \ref{tab:table2} in the original paper, compares different models based on four evaluation metrics (Sensitivity, Specificity, Accuracy and MCC) for 
    "all seven substrate-specific transporter classes". Also, under "Data Compilation" section, it has been mentioned that the five-fold 
    cross-validation has been applied to "1,380 proteins in the main dataset" that is the total number of sequences available across 
    all 8-classes being put together.

    So, to clear up any doubts, for each feature we generated the ".csv" file for both 7-class (all seven transporter classes) and 8-class 
    (all seven classes plus non-transporters) based models using \git{extractFeature.py}{extractFeature.py} program. We then run the 
    classifier on both models to find out which model would produce the closest results to the ones being provided in the paper.
    
    
    \myparagraph{\small Amino Acid Composition (AAC)}
    
    The Amino Acid (Monopeptide) Composition is the number of amino acids of each type normalized with the total number of 
    residues \cite{gromiha2010protein}. The percentage of each amino acid is calculated using following formula where $i$ 
    represents one of the 20 standard amino acids \cite{mishra2014prediction} :
    \begin{equation}
        \text{percentage of amino acid (i)} = \frac {\text{total number of amino acid (i)}} {\text{total number of amino acids in protein}} * 100
    \end{equation} 
    
    \myparagraph{\small Dipeptide Composition (DPC)}
    
    The composition of dipeptides is a measure to quantify the preference of amino acid residue pairs in a sequence 
    \cite{gromiha2010protein}. The percentage of each dipeptide is calculated using following formula where $i$ 
    can be any dipeptide of 400 possible dipeptides. \cite{mishra2014prediction}:
    \begin{equation}
        \text{percentage of dipeptide (i)} = \frac {\text{total number of dipeptide (i)}} {\text{total number of dipeptides in protein}} * 100
    \end{equation}


    \myparagraph{\small Physico-Chemical Composition (PHC)}
    
    The physico-chemical composition is the composition of the physico-chemical class residues in each protein sequence 
    \cite{mishra2014prediction}. Percentage composition of following 11 physico-chemical properties is the input to our 
    SVM models for this feature \cite{mishra2014prediction, kumar2008copid}:\\
    
    Aliphatic (I, L, V), Neutral (D, E, R, K, Q, N), Aromatic (F, H, W, Y),
    Hydrophobic (C, V, L, I, M, F, W), Charged (D, E, K, H, R), Positively charged (H, K, R), 
    Negatively charged (D, E), Polar (D, E, R, K, Q, N),
    Small (E, H, I, L, K, M, N, P, Q, V), Large (F, R, W, Y), Tiny (A, C, D, G, S, T).

    \myparagraph{\small Biochemical Composition (AAindex)}
    
    An amino acid index (AAindex) is a set of 20 numerical values (for 20 standard amino acids) representing various physico-chemical 
    and biochemical properties of amino acids which are subsets of AAIndex database. \cite{aaindex} For this study, 49 physical, 
    chemical, energetic, and conformational properties of amino acids are selected as an input feature to the SVM model. These properties 
    has been used to study protein folding and stability and transporter classification.
    \cite{zavaljevski2002support, gromiha1999importance, gromiha2006statistical} (The 49 selected properties and the details is 
    available in our \git{aaindex.py}{GitHub}) The average of each property for each protein sequence is calculated using following formula:
    \begin{equation}
        \text{Amino Acid Index (i)} = \frac {\sum_{j=1}^{n} AAind_{ij}} {\text{n}}
    \end{equation}
    where $n$ is the length of the protein sequence, $AAind_{i}$ is the $ith$ biochemical property, $AAind_{ij}$ is the value of $ith$ 
    biochemical property for the $jth$ amino acid in the sequence and $\sum_{j=1}^{n} AAind_{ij}$ is the sum of $ith$ property for all the
    $n$ amino acids in a protein sequence.

    \myparagraph{\small Position-specific scoring matrix (PSSM) profile}
        \myparagraph{iteration for PSSM??}

    
  
    \tbm{link to code?}

    % =======================================================================================================================
    % =======================================================================================================================
    
    \subsection{The Model}
    After extracting all the features mentioned above and using the available information from the paper, 
    the first model was built on 7 class-based amino acid composition (\pa{AAC}) feature. But the  reproduced 
    results had considerable differences with the ones from the paper.  
    Table \ref{tab:table1} shows the accuracy and MCC from  the first coded model compared to the original results.\\

    In order to find the probable missing problem-related parameters from the report, 
    we coded different models on the AAC feature to observe how different instances of those 
    parameters could affect the predicted results and consequently the reported metrics.\\

    Considering the imbalanced nature of the problem, we experimented on different instances of 
    a single dataset (Normal, Shuffled, Downsampled), two instances of the main dataset (7 vs 8 class-based sets), 
    two libraries for support vector machine classification algorithm (SVMLight and Scikit-learn), two different settings 
    for Gamma and Cost, three different evaluation models (before voting, thresholding and after voting) and 
    two averaging techniques (Micro vs Macro) to find the best model that could provide the closest results to the reported ones. 
    Table \ref{tab:table4} and Table \ref{tab:table5} shows the average Accuracy, Sensitivity, Specificity and MCC 
    results for all those models mentioned above.\\

    We then used the settings of our best model, to build new computational models for the rest of the features. 
    Table \ref{tab:table6} shows the average sensitivity, specificity, accuracy, and MCC of our best model 
    along with its original published results for each feature. The order and combination of the features are the same 
    as the ones mentioned in Table1 of the paper.\\


    \subsubsection{Basic Scikit-Learn based model}
    The basic model is completely based on Scikit-Learn library. In this model, for any involved 
    computational, statistical or evaluation task (i.e. train/test data split, 5-fold cross-validation, grid search, etc.) 
    we called on a built-in function from the Scikit-Learn library through our pipeline.\\

    \myparagraph{Construction of the main dataset}

    In this model, for the main dataset, we experimented on two different sets: the transporter proteins set 
    (7 classes) and the combination of the transporter and non-transporter proteins set (8 classes). 
    The 7 class-based dataset contains amino acid composition vectors for 780 transporter proteins classified into 
    seven substrate-specific classes(amino acid, anion, cation, electron, others, protein and sugar). 
    The 8-class based dataset contains 1380 amino acid composition vectors for 600 non-transporter proteins and 
    780 transporter proteins mentioned above.\\

    For the datasets mentioned above, we then generated three different instances of each (Normal, Shuffled, Downsampled) 
    to observe how changes in order and number of the vectors could affect the results in an imbalanced dataset.\\
    
    The Normal set is the dataset in its original order. For example, the normal version of the 7 class-based set, 
    contains 780 vectors. Each vector has 21 values (20 from the extracted amino acid composition feature and one for the label) 
    and represents a transporter in that dataset. The set starts with all 70 vectors of the amino acid class 
    followed by 60 anions, 260 cations, 60 electrons, 70 proteins/mRNAs, 60 sugar and 200 transporters from other class. 
    The shuffled set is the shuffled version of the Normal set mentioned above. The transporter vectors in the set are 
    shuffled by the random seed of 2.\\

    The down-sampled set contains 60 sequences from each class (number of records in our anion class with 
    the least amount of sequences) being taken randomly from that specific class. These subsamples were then put together in 
    a random order (using the random seed mentioned above) to create a balanced final set. The final number of records 
    are 420 for 7-class based dataset and 480 for the 8 class-based set.\\

    
    \myparagraph{Cross Validation}

    Cross-validation is a model validation technique for assessing how the results of 
    statistical analysis (a newly developed model) would generalize to an independent dataset. 
    The goal of the cross-validation is to limit problems like overfitting and underfitting. 
    For this problem, we applied 5-fold cross-validation to each one of those instances mentioned above using 
    "KFold" function with the "n\_splits" parameter value of 5 from Scikit-Library.\\

   
    \myparagraph{Classification Algorithm}
    
    The support vector machine (SVM) is a universal machine learning approximator based on the structural risk minimization (SRM) 
    principle of statistical learning theory \cite{vapnik1995}. The algorithm could be applied to both regression and 
    classification problems. This technique is particularly attractive to biological sequence analysis because many biological 
    problems involve high-dimensional noisy data, for which SVMs are known to behave well 
    compared to the other ones.\cite{zavaljevski2002support}\\

    For SVM classification algorithm, we tried both "OneVsRestClassifier" and "SVC" with the "decision\_function\_shape" 
    being set to "ovr" (for One against all classification) from scikit-learn library. The results from both functions 
    were very close. So, we picked "SVC" with “rbf” kernel and we set the "decision\_function\_shape" to "ovr" (one versus rest) 
    for our experiment.\\

    For Gamma and Cost values, the provided value range from the paper is 1-e-5 to 10 for the Gamma value and 1 to 4 for 
    the Cost value. So, using the "Grid Search" function from Scikit-learn library, we searched for the best parameters 
    on each one of those six instances (normal, shuffled and down-sampled for 7 and 8 class based datasets) and we 
    used the correspondent results for each instance.\\
  

    \myparagraph{Evaluation}
    To evaluate the models' performances, we used accuracy and Matthews correlation coefficient (MCC) metrics.
    Table \ref{tab:table1} results are the product of "accuracy\_score" and "matthews\_corrcoef" functions from scikit-learn 
    library in our pipeline.\\

    \myparagraph{Results}
        
    Table \ref{tab:table1} shows the average accuracy and matthew's correlation coefficient (\pa{MCC}) values for different 
    basic scikit-learn based models for all seven and eight substrate-specific classes for amino acid composition (AAC) feature. 
    The correspondent Gamma and Cost values for each model, is the result of a "Grid Search" on that model.\\

    The best performance is achieved on the model with 8 shuffled classes in the main dataset (accuracy:53.76, MCC:0.34). 
    Overall, the models with 8 classes of sequences in the dataset, show better performances compared to the 7 class-based sets. 
    Because there is more information available (780 vectors compared to 1380) for the algorithm through the training phase.\\

    Considering the imbalance nature of our data, when we divide the normal instance of the main dataset into 5 different folds, 
    in each fold, there will be at least one class which is present in the test set but not in the test set. That means, 
    the algorithm should decide on records of data that it hasn’t seen before through the training phase.\\

    In the case of the shuffled instance of the main dataset, we can be sure that in the train and test sets of 
    each fold, there would be a portion of each class available. So, the algorithm will be trained on all the 
    transporter or non-transporter classes. But the number of vectors from each class present in either train or 
    test set would be different in each fold.\\

    In the case of the downsampled instance of the main dataset, we have created a balanced set with an equal number of 
    sequences from each class. The problem is that we lose some information throughout the process. We would have 420 items 
    for 7 class-based sets (compared to 780 items before downsampling) and 480 items for 8 class-based sets 
    (compared to the 1380 records before downsampling). But, we can be sure that there will be an equal number of sequences 
    from each class present in the train and the test set of each fold.\\

    
    % Table 1
    
\begin{table}[ht]
    \centering
    \begin{tabular}{|L | V |V V V V g g g V V V V | V |}
        \hline
        \multicolumn{14}{|g|}{Scikit-learn prediction-based models}\\
        \hline
        \multicolumn{1}{|l|}{\multirow{2}{*}{\footnotesize{Original Results}}}
        &
        \multicolumn{3}{g}{Accuracy} & \multicolumn{3}{g}{Sensitivity} &
        \multicolumn{3}{g}{Specificity} & \multicolumn{4}{g|}{MCC}\\
        \cline{2-14}&
        \multicolumn{3}{C}{73.74} & \multicolumn{3}{C}{74.65} &
        \multicolumn{3}{C}{73.22} & \multicolumn{4}{C|}{0.46} \\
        \hline
        
        \cline{1-12}\multicolumn{1}{|g}{}&
        
        \multicolumn{1}{|g|}{\footnotesize{Dist}}&
        \multicolumn{4}{g} {\footnotesize{7 class-based models}}&
        \multicolumn{3}{g}{}&
        \multicolumn{4}{g|} {\footnotesize{8 class-based models}}&
        \multicolumn{1}{g|}{\footnotesize{Dist}}\\
        
        \cline{3-6}\cline{10-13}\multicolumn{1}{|g}{}&
        \multicolumn{1}{|g|}{\footnotesize{ance}}&
        \multicolumn{1}{g}{acc}&\multicolumn{1}{g}{sens}&
        \multicolumn{1}{g}{spec}&\multicolumn{1}{g}{mcc}&
        \multicolumn{3}{g}{}&
        \multicolumn{1}{g}{acc}&\multicolumn{1}{g}{sens}&
        \multicolumn{1}{g}{spec}&\multicolumn{1}{g|}{mcc}&\multicolumn{1}{g|}{\footnotesize{ance}}\\

        \hline

        \multicolumn{1}{|l|}{\multirow{6}{*}{\footnotesize{Prediction-based}}}
        
        & 0.70 & 76.88 & 19.10 & 86.51 & 0.05 &    s&&s                & 84.58 & 38.33 & 91.19 & 0.29 & 0.45 \\
        & 0.39 & 83.66 & 42.86 & 90.47 & 0.33 &    d&\small{Micro}&d   & 84.79 & 39.16 & 91.31 & 0.30 & 0.44 \\
        & \mrkm{0.36}{84.43}{45.51}{90.91}{0.36}&    sh&&sh              & \mrkm{88.71}{54.85}{93.54}{0.48}{0.32} \\

        
        \cline{2-5}\cline{7-9}\cline{11-14}
        
        & 0.81 & 76.88 & 6.69 & 81.70 & 0.01 &    s&&s                & 84.58 & 76.82 & 86.79 & 0.03 & 0.81 \\
        & 0.39 & 83.67 & 42.85 & 90.47 & 0.32 &    d&\small{Macro}&d   & 84.79 & 39.16 & 91.31 & 0.29 & 0.44 \\
        & 0.39 & 85.27 & 43.31 & 90.33 & 0.35 &    sh&&sh              & 88.71 & 35.32 & 91.94 & 0.29 & 0.48 \\
        
        \hline\hline
        
        \multicolumn{14}{|c|} {\footnotesize{
            S, D and SH are sorted, down-sampled and shuffled instances of the main dataset.
        }}\\
        \multicolumn{14}{|c|} {\footnotesize{
            Acc: Accuracy, Sens: Sensitivity, Spec: Specificity, Mcc: Matthews correlation coefficient
        }}\\

        \hline
        
       

    \end{tabular}
    \captionsetup{font=small,width=14cm}
    \caption{The average sensitivity, specificity, accuracy, and MCC values 
    for scikit-learn prediction-based models for amino acid composition (AAC).}
    \label{tab:table1}
    
\end{table}

    \subsubsection{One vs All model}
    Considering the differences in between our results from the basic Scikit-Learn model and the ones of 
    the paper on AAC feature, we tried a One vs All model to improve the results. In this model, we coded all 
    the phases without making any call on functions from different libraries (except for the classification algorithm) 
    to gain more insight on the other problem-related parameters that could affect the model performance.\\

    \myparagraph{Construction of the main dataset}
    
    Trough this model and for the main dataset, we experimented on two following sets: the transporter proteins set 
    (7 classes) and the combination of the transporter and non-transporter proteins set (8 classes). 
    The 7 class-based dataset contains amino acid composition vectors for 780 transporter proteins classified into 
    seven substrate-specific classes(amino acid, anion, cation, electron, others, protein and sugar). 
    The 8-class based dataset contains 1380 amino acid composition vectors for 600 non-transporter proteins and 
    780 transporter proteins mentioned above.\\

    Considering the imbalanced nature of our dataset, for each set mentioned above, we then generated 
    three different instances (Normal, Shuffled, Downsampled) to observe how changes in order 
    and number of the vectors could affect the results in that setting.\\

    For the normal insrance of the dataset, we classified all sequences under seven (for the 7 class-based sets) or 
    eight (for 8 class-based sets) categories. We then applied cross-validation to each category where we created 
    5 folds with its correspondent train and test sets for each one of those classes. For each final fold 
    (i.e. final dataset's first fold), we attached the related fold from all the classes (i.e. first fold from 
    the amino acid, anion, cation, etc.) This way, considering the imbalanced nature of our dataset, for each fold, 
    we could make sure that we have an equal portion of each class in train (4/5th of the class sequences) and test 
    (1/5th of the class sequences) sets.\\

    For the shuffled set, we first, shuffled the main dataset using random seed of 2. 
    We then applied 5-fold cross-validation and created the correspondent train and test sets.\\

    The down-sampled set contains 60 sequences from each class (number of records in our anion class with 
    the least amount of sequences) being taken randomly from that specific class. These subsamples were then put together in 
    a random order (using the random seed mentioned above) to create a balanced final set. Like the normal set, we then 
    classified them under seven (for the 7 class-based sets) or eight (for 8 class-based sets) categories, applied 
    5-fold cross-validation to each one and then attached the correspondent folds together to create the final folds.
    The final number of records are 420 for 7-class based dataset and 480 for the 8 class-based set.\\

    For each one of those 6 datasets mentioned above (normal, shuffled and down-sampled for 7 and 8 class based sets), 
    and for each fold, we transformed the set into "One against All" format. In this format, in each fold, 
    from each set (train or test), we produced seven (for the 7 class-based sets) or eight (for 8 class-based sets) 
    sets in a way that, for each class (i.e. anion transporters), we labeled the records from that class (i.e. anion) 
    with "+1" and the rest of the classes with "-1". So, for example, for fold one of 7-class based dataset, 
    we generated 7 new sets through which for example for amino acid class, we would have 70 classes 
    (amino acid transporters) with "+1" label and 710 other records with "-1" label. 
    Same thing has been done for the rest of the classes in the same fold. 
    A sample generated set is available at the paper’s GitHub. \\

    \myparagraph{Classification}

    \myparagraph{\footnotesize{Choice of classification algorithm}}
    In this paper, for the classification algorithm, we both experimented with SVMLight software, 
    and the Scikit-Learn's "SVC" classifiers in our models. \\

    \myparagraph{\footnotesize{Gamma and Cost}}
    For the Gamma and Cost, we once tried classification with
    the best values for all classes (being chosen through Grid Search) and the other time for 
    each class we assigned a different Gamma and Cost values specific to that class (being chosen by applying Grid Search to 
    that specific class instead of the whole dataset).\\

    So, for each dataset (7 vs 8 class based dataset), we tried 6 different 
    scenarios(i.e. shuffled 7-class set wiht one Gamma and Cost values for all classes versus the same set with values 
    different for each class.)\\

    \myparagraph{\footnotesize{Before Voting}}
    We also considered 3 different following approaches for producing the results:\\ 
    First, for each fold, for each class (i.e. amino acids vs all others), we evaluated the classification 
    results for that class (accuracy, sensitivity, specificity, MCC) and then we averaged through the results for 
    all classes in a fold, for the results for that fold. we then averaged through all 5 folds, for the final results. In the 
    Tables for results (2,3,4,5,6,7) we referred to this one as "Before Voting".\\
    
    \myparagraph{\footnotesize{Thresholding}}
    Compared to the original results, the results we produced above, were close but the difference in between thr metrics were not 
    in the same range.(i.e. higher values for accuracy and specificity close values for MCC and lower values for sensitivity). 
    For the next one, we went through the same process, but we applied a threshold value to all the classes (and through 
    all folds) for balancing out the results (We referred to this model as "Thresholding" through the tables' of results)\\

    After classification (for both scikit-learn and SVMLight) and through each fold, for each class, we produced probability values 
    for each row of values. In the Normal mode, we considered the ones less than 0 as class -1 and the ones above 
    that value as +1. and from that, we calculated the 4 elements of confusion matrix (TP, FP, TN, FN) and then we calculated the 
    metrics from those. For thresholding model, we considered the (0 - threshold) as the border value for classification in between 
    positive and negative results(+1 vs -1).\\

    For Scikit-learn, there is possibility where the SVC algorithm could provide you with both the probability or the 
    predicted label(+1 or -1). For this experiment and for each classification case, we produced both results. For the ones with 
    the prediction label of "-1" we multiplied the probability by -1 and then save it to the file. So, ??\\
    
    \myparagraph{\footnotesize{After Voting}}
    Through the third model, instead of averaging in between all the classes results for each fold, we did voting through each fold 
    (in between 7 classifiers) and then using the final results, we evaluated the results for all 7 classes against the original labels 
    for that fold and then produced the metrics' values(accuracy, sensitivity, specificity, MCC). we then averaged through all 5 folds 
    for the final values(We referred to this model as "After voting" through the tables' of results).\\

    

    \myparagraph{Evaluation}
        
        In this study, sensitivity, specificity, accuracy and Matthew's correlation coefficient (MCC) is used to measure the 
        prediction performance. For evaluation, these metrics are calculated for each test dataset through the five-fold cross validation.
        For the final value, the parameters computed from each subset were averaged across all five subsets . TP, FP, TN, FN are 
        true positives, false positives, true negatives and false negatives, respectively. \\

        Sensitivity is a measure of the proportion of actual positive cases that got predicted as positive (or true positive). 
        Sensitivity is also termed as Recall (and in this case is the percentage of transporters 
        that were correctly predicted as transporters) and is calculated as follow:
        \begin{equation}
            \text{Sensitivity} = \frac {\text{TP}}{\text{TP + FN}} * 100
        \end{equation}

        Specificity is defined as the proportion of actual negatives, which got predicted as the negative (or true negative) and 
        (the percentage of non-transporters that were correctly predicted as non-transport proteins) is computed using following formula:
        \begin{equation}
            \text{Specificity} = \frac {\text{TN}}{\text{TN + FP}} * 100
        \end{equation}

        Accuracy is the overall percentage of the correctly predicted classes (either true positive or true negative) to all the 
        predictions. In this case it's the overall percentage of transporters and non-transporters 
        that were correctly predicted and calculated using following formula:
        \begin{equation}
            \text{Accuracy} = \frac {\text{TP + TN}}{\text{TP + FP + TN + FN}} * 100
        \end{equation}

        The Matthews correlation coefficient (MCC) is used in machine learning as a measure of the quality of binary 
        classifications, introduced by biochemist Brian W. Matthews in 1975. The coefficient takes into account true 
        and false positives and negatives and is generally regarded as a balanced measure which can be used even if the classes 
        are of very different sizes. The MCC is in essence a correlation coefficient between the observed and predicted binary 
        classifications. it returns a value between -1 and +1. A coefficient of +1 represents a perfect prediction, 0 no better than 
        random prediction and -1 indicates total disagreement between prediction and observation. It is computed using the following 
        formula: 

        \begin{equation}
            \text{MCC} = \frac {\text{(TP * TN) - (FN * FP)}}{\sqrt{(TP + FN) * (TN + FP) * (TP + FP) * (TN + FN)}}
        \end{equation}

        \myparagraph{\footnotesize{Micro vs Macro}}
        Micro- and macro-averages (for any metric) will compute slightly different things, and thus their interpretation differs. 
        A macro-average will compute the metric independently for each class and then take the average (hence treating 
        all classes equally), whereas a micro-average will aggregate the contributions of all classes to compute the average metric.\\

        For each model, whenever we averaged through results (in-between all classes to produce results of one fold or 
        in-between all folds for the final results) we tried both Micro and Macro averaging. 

        \myparagraph{Results}
        Regarding the number of classes in the dataset, for 7-class based dataset we had 780 sequences and for 8-class based set, 
        we had 1380 sequences in our set. Through all the models, The results are better for 8-class set as we have more information 
        available. But the noticeable difference (because of which we decided to go with 8-class based set) is on the MCC metric. 
        The best MCC value achieved for 7-class based dataset was 0.38 using SVMLight and 0.39 using the Scikit-Learn SVC classifier. 
        But for 8-class based set, the best results are 0.49 for SVMLight and 0.48 for Scikit-Learn classifier, which is higher than 
        original value (0.46). and through introducing a threshold, we could get it close to that value.\\

        Regarding the different kind of datasets (normal, shuffled, down-sampled), the results are very close for normal and shuffled 
        sets but for down-sampled sets the sensitivity and MCC are lower than those of the shuffled and normal sets. The reason is, 
        in normal set, we get the the same portion of each class in the train/test sets in each fold, althogh in shuffled version, the 
        we dont have the same portion of the set from each class in each fold, but the difference is not very big. so the results are not 
        than different,.  in down-sampled version we loose lots of information (480 compared to 1380) but its balanced. so, beside 
        the accuracy which is close, the others are different .... because we loose lots of true positives on majority classes 
        (Cation, non-transporter)


        Regarding the Gamma and Cost Values, For each class, once we assigned values specific to that class, 
        and the other time we used one global Gamma and Cost value for all classes. The results are very close, but overall
        through first scenario, we achieved better results for the mterics as each class had a better True Positive rate using 
        values being accustomed to its data model.\\

        Regarding SVMLight vs Scikit-learn, for accuracy the results are close. SVMLight provides better results for sensitivity, 
        MCC and overall a more balanced results. but scikit produces lower sensitivity and higher specificity compared to SVM light.\\

        Regarding Micro vs Macro averaging, through results, accuracy, sensitivity and 
        specificity results are really close (and sometimes even the same) for all models, the only noticeable 
        difference is on MCC value (When it is not a down-sampled set). 
        [Table ... shows that] on all models, Macro averaging values could never get close to the 
        original values of the paper (on MCC) but Micro averaging produced the expected results on MCC.\\

        Overall compared to original results (\pa{Accuracy: 73.74, Sensitivity: 74.65, 
        Specificity: 73.22 , MCC: 0.46}) the best results were achieved oin following scenarios:\\

        On 7-class based datasets, the best results are achieved by using the threshold model on Normal version of a dataset when 
        assigning different Gamma and Cost values for each class when using SVMLight for classification 
        (\pa{Accuracy: 74.04, Sensitivity: 74.74, Specificity: 73.93 , MCC: 0.36}). 
        The only problem is with the MCC value with the a difference of 0.10.

        \myparagraph{8-class}
        For 8-class based datasets, the results with highest values are reproduced by SVM classifier with different Gamma and Cost 
        value for each class when Micro averaging the folds' results on 
        "Before Voting" model being applied to either normal or shuffle version of the the dataset.
        (\pa{Accuracy: 87.33, Sensitivity: 64.92, Specificity: 90.53 , MCC: 0.49}). Through this model, although all the metrics' 
        values are above the reported values, but, the sensitivity was lower than than reported one compared to the rest of the 3.
        so, the best result which provides us with values closer to the reported one which balances out specificity and sensitivity, 
        is when you apply the same algorithm (with the same Gamma and cost strategy) to thresholding model, on normal or shuffled sets 
        and applying micro averaging. (\pa{Accuracy: 79.98, Sensitivity: 79.78, Specificity: 80.01 , MCC: 0.44}).\\

% =======================================================================================================================
% =======================================================================================================================