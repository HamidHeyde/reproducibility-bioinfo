\section{Materials and Methods}
    ==> Normally, in machine learning related problems, the solution is divided into 4 phases, 1. putting 
        together a dataset, 2. data preprocessing, 3. model building and 4. model evaluation\\
        Some discussion over how the whole process works.(here or in introduction)\\

    \subsection{Dataset}

    The dataset contains 780 transporter proteins classified into 7 substrate specific classes (70 amino acid transporters, 
    60 anion transporters, 260 cation transporters, 60 electron transporters, 70 protein/mRNA transporters, 60 sugar 
    transporters, 200 other transporters) and one none-transporter class (600 proteins) for a total of 1,380 protein 
    sequences which are available at  \trssp{?dowhat=Datasets}{TrSSP} website.
    
    For the purpose of our study, we coded a program (\git{download.py}{download.py}) to download all the sequences from the 
    \ncbi{protein}{NCBI} database using the sequence accession number from TrSSP website. 
    Considering the fact that some sequences could be updated through time, we checked all the \git{dataset/trainTest/}{downloaded} 
    sequences against the originals to make sure all of our sequences match those of the TrSSP. We then updated the 
    modified sequences and their accession numbers accordingly.

    \subsection{Feature extraction}

    The authors \cite{mishra2014prediction} have extracted five features out of the dataset sequences. They have then built-up different Support Vector Machine based 
    computational models using a combination of these features. Following the descriptions from the paper, we generated all the 
    \git{features/}{features} in Comma Separated Values (.csv) format adding one more column for the labels.

    Table \ref{tab:table2} in the original paper, compares different models based on four evaluation metrics (Sensitivity, Specificity, Accuracy and MCC) for 
    "all seven substrate-specific transporter classes". Also, under "Data Compilation" section, it has been mentioned that the five-fold 
    cross-validation has been applied to "1,380 proteins in the main dataset" that is the total number of sequences available across 
    all 8-classes being put together.

    So, to clear up any doubts, for each feature we generated the ".csv" file for both 7-class (all seven transporter classes) and 8-class 
    (all seven classes plus non-transporters) based models using \git{extractFeature.py}{extractFeature.py} program. We then run the 
    classifier on both models to find out which model would produce the closest results to the ones being provided in the paper.
    
    
    \myparagraph{\small Amino Acid Composition (AAC)}
    
    The Amino Acid (Monopeptide) Composition is the number of amino acids of each type normalized with the total number of 
    residues \cite{gromiha2010protein}. The percentage of each amino acid is calculated using following formula where $i$ 
    represents one of the 20 standard amino acids \cite{mishra2014prediction} :
    \begin{equation}
        \text{percentage of amino acid (i)} = \frac {\text{total number of amino acid (i)}} {\text{total number of amino acids in protein}} * 100
    \end{equation} 
    
    \myparagraph{\small Dipeptide Composition (DPC)}
    
    The composition of dipeptides is a measure to quantify the preference of amino acid residue pairs in a sequence 
    \cite{gromiha2010protein}. The percentage of each dipeptide is calculated using following formula where $i$ 
    can be any dipeptide of 400 possible dipeptides. \cite{mishra2014prediction}:
    \begin{equation}
        \text{percentage of dipeptide (i)} = \frac {\text{total number of dipeptide (i)}} {\text{total number of dipeptides in protein}} * 100
    \end{equation}


    \myparagraph{\small Physico-Chemical Composition (PHC)}
    
    The physico-chemical composition is the composition of the physico-chemical class residues in each protein sequence 
    \cite{mishra2014prediction}. Percentage composition of following 11 physico-chemical properties is the input to our 
    SVM models for this feature \cite{mishra2014prediction, kumar2008copid}:\\
    
    Aliphatic (I, L, V), Neutral (D, E, R, K, Q, N), Aromatic (F, H, W, Y),
    Hydrophobic (C, V, L, I, M, F, W), Charged (D, E, K, H, R), Positively charged (H, K, R), 
    Negatively charged (D, E), Polar (D, E, R, K, Q, N),
    Small (E, H, I, L, K, M, N, P, Q, V), Large (F, R, W, Y), Tiny (A, C, D, G, S, T).

    \myparagraph{\small Biochemical Composition (AAindex)}
    
    An amino acid index (AAindex) is a set of 20 numerical values (for 20 standard amino acids) representing various physico-chemical 
    and biochemical properties of amino acids which are subsets of AAIndex database. \cite{aaindex} For this study, 49 physical, 
    chemical, energetic, and conformational properties of amino acids are selected as an input feature to the SVM model. These properties 
    has been used to study protein folding and stability and transporter classification.
    \cite{zavaljevski2002support, gromiha1999importance, gromiha2006statistical} (The 49 selected properties and the details is 
    available in our \git{aaindex.py}{GitHub}) The average of each property for each protein sequence is calculated using following formula:
    \begin{equation}
        \text{Amino Acid Index (i)} = \frac {\sum_{j=1}^{n} AAind_{ij}} {\text{n}}
    \end{equation}
    where $n$ is the length of the protein sequence, $AAind_{i}$ is the $ith$ biochemical property, $AAind_{ij}$ is the value of $ith$ 
    biochemical property for the $jth$ amino acid in the sequence and $\sum_{j=1}^{n} AAind_{ij}$ is the sum of $ith$ property for all the
    $n$ amino acids in a protein sequence.

    \myparagraph{\small Position-specific scoring matrix (PSSM) profile}
        \myparagraph{iteration for PSSM??}

    
  
    \tbm{link to code?}

    % =======================================================================================================================
    % =======================================================================================================================
    
    \subsection{The Model}
    After extracting all the features mentioned above and using the available information from the paper, 
    the first model was built on 7 class-based amino acid composition (\pa{AAC}) feature. But the  reproduced 
    results had considerable differences with the ones from the paper.  
    Table \ref{tab:table1} shows the accuracy and MCC from  the first coded model compared to the original results.\\

    In order to find the probable missing problem-related parameters from the report, 
    we coded different models on the AAC feature to observe how different instances of those 
    parameters could affect the predicted results and consequently the reported metrics.\\

    Considering the imbalanced nature of the problem, we experimented on different instances of 
    a single dataset (Normal, Shuffled, Downsampled), two instances of the main dataset (7 vs 8 class-based sets), 
    two libraries for support vector machine classification algorithm (SVMLight and Scikit-learn), two different settings 
    for Gamma and Cost, three different evaluation models (before voting, thresholding and after voting) and 
    two averaging techniques (Micro vs Macro) to find the best model that could provide the closest results to the reported ones. 
    Table \ref{tab:table4} and Table \ref{tab:table5} shows the average Accuracy, Sensitivity, Specificity and MCC 
    results for all those models mentioned above.\\

    We then used the settings of our best model, to build new computational models for the rest of the features. 
    Table \ref{tab:table6} shows the average sensitivity, specificity, accuracy, and MCC of our best model 
    along with its original published results for each feature. The order and combination of the features are the same 
    as the ones mentioned in Table1 of the paper.\\


    \subsubsection{Basic Scikit-Learn based model}
    The basic model is completely based on Scikit-Learn library. In this model, for any involved 
    computational, statistical or evaluation task (i.e. train/test data split, 5-fold cross-validation, grid search, etc.) 
    we called on a built-in function from the Scikit-Learn library through our pipeline.\\

    \myparagraph{Construction of the main dataset}

    In this model, for the main dataset, we experimented on two different sets: the transporter proteins set 
    (7 classes) and the combination of the transporter and non-transporter proteins set (8 classes). 
    The 7 class-based dataset contains amino acid composition vectors for 780 transporter proteins classified into 
    seven substrate-specific classes(amino acid, anion, cation, electron, others, protein and sugar). 
    The 8-class based dataset contains 1380 amino acid composition vectors for 600 non-transporter proteins and 
    780 transporter proteins mentioned above.\\

    For the datasets mentioned above, we then generated three different instances of each (Normal, Shuffled, Downsampled) 
    to observe how changes in order and number of the vectors could affect the results in an imbalanced dataset.\\
    
    The Normal set is the dataset in its original order. For example, the normal version of the 7 class-based set, 
    contains 780 vectors. Each vector has 21 values (20 from the extracted amino acid composition feature and one for the label) 
    and represents a transporter in that dataset. The set starts with all 70 vectors of the amino acid class 
    followed by 60 anions, 260 cations, 60 electrons, 70 proteins/mRNAs, 60 sugar and 200 transporters from other class. 
    The shuffled set is the shuffled version of the Normal set mentioned above. The transporter vectors in the set are 
    shuffled by the random seed of 2.\\

    The down-sampled set contains 60 sequences from each class (number of records in our anion class with 
    the least amount of sequences) being taken randomly from that specific class. These subsamples were then put together in 
    a random order (using the random seed mentioned above) to create a balanced final set. The final number of records 
    are 420 for 7-class based dataset and 480 for the 8 class-based set.\\

    
    \myparagraph{Cross Validation}

    Cross-validation is a model validation technique for assessing how the results of 
    statistical analysis (a newly developed model) would generalize to an independent dataset. 
    The goal of the cross-validation is to limit problems like overfitting and underfitting. 
    For this problem, we applied 5-fold cross-validation to each one of those instances mentioned above using 
    "KFold" function with the "n\_splits" parameter value of 5 from Scikit-Library.\\

   
    \myparagraph{Classification Algorithm}
    
    The support vector machine (SVM) is a universal machine learning approximator based on the structural risk minimization (SRM) 
    principle of statistical learning theory \cite{vapnik1995}. The algorithm could be applied to both regression and 
    classification problems. This technique is particularly attractive to biological sequence analysis because many biological 
    problems involve high-dimensional noisy data, for which SVMs are known to behave well 
    compared to the other ones.\cite{zavaljevski2002support}\\

    For SVM classification algorithm, we tried both "OneVsRestClassifier" and "SVC" with the "decision\_function\_shape" 
    being set to "ovr" (for One against all classification) from scikit-learn library. The results from both functions 
    were very close. So, we picked "SVC" with “rbf” kernel and we set the "decision\_function\_shape" to "ovr" (one versus rest) 
    for our experiment.\\

    For Gamma and Cost values, the provided value range from the paper is 1-e-5 to 10 for the Gamma value and 1 to 4 for 
    the Cost value. So, using the "Grid Search" function from Scikit-learn library, we searched for the best parameters 
    on each one of those six instances (normal, shuffled and down-sampled for 7 and 8 class based datasets) and we 
    used the correspondent results for each instance.\\
  

    \myparagraph{Evaluation}
    To evaluate the models' performances, we used accuracy and Matthews correlation coefficient (MCC) metrics.
    Table \ref{tab:table1} results are the product of "accuracy\_score" and "matthews\_corrcoef" functions from scikit-learn 
    library in our pipeline.\\

    \myparagraph{Results}
        
    Table \ref{tab:table1} shows the average accuracy and matthew's correlation coefficient (\pa{MCC}) values for different 
    basic scikit-learn based models for all seven and eight substrate-specific classes for amino acid composition (AAC) feature. 
    The correspondent Gamma and Cost values for each model, is the result of a "Grid Search" on that model.\\

    The best performance is achieved on the model with 8 shuffled classes in the main dataset (accuracy:53.76, MCC:0.34). 
    Overall, the models with 8 classes of sequences in the dataset, show better performances compared to the 7 class-based sets. 
    Because there is more information available (1380 vectors compared to 780) for the algorithm through the training phase.\\

    Considering the imbalance nature of our data, when we divide the normal instance of the main dataset into 5 different folds, 
    in each fold, there will be at least one class which is present in the test set but not available in the train set. That means, 
    the algorithm should decide on records of data that it hasn’t seen before through the training phase.\\

    In the case of the shuffled instance of the main dataset, we can be sure that in the train and test sets of 
    each fold, there would be a portion of each class available. So, the algorithm will be trained on all the 
    transporter or non-transporter classes. But the number of vectors from each class present in either train or 
    test set would be different in each fold.\\

    In the case of the downsampled instance of the main dataset, we have created a balanced set with an equal number of 
    sequences from each class. The problem is that we lose some information throughout the process. We would have 420 items 
    for 7 class-based sets (compared to 780 items before downsampling) and 480 items for 8 class-based sets 
    (compared to the 1380 records before downsampling). But, we can be sure that there will be an equal portion of sequences 
    from each class present in the train and the test set of each fold (4/5th of the total sequences of a class 
    for the train set and 1/5th for the test set).\\

    
    % Table 1
    
\begin{table}[ht]
    \centering
    \begin{tabular}{|L | V |V V V V g g g V V V V | V |}
        \hline
        \multicolumn{14}{|g|}{Scikit-learn prediction-based models}\\
        \hline
        \multicolumn{1}{|l|}{\multirow{2}{*}{\footnotesize{Original Results}}}
        &
        \multicolumn{3}{g}{Accuracy} & \multicolumn{3}{g}{Sensitivity} &
        \multicolumn{3}{g}{Specificity} & \multicolumn{4}{g|}{MCC}\\
        \cline{2-14}&
        \multicolumn{3}{C}{73.74} & \multicolumn{3}{C}{74.65} &
        \multicolumn{3}{C}{73.22} & \multicolumn{4}{C|}{0.46} \\
        \hline
        
        \cline{1-12}\multicolumn{1}{|g}{}&
        
        \multicolumn{1}{|g|}{\footnotesize{Dist}}&
        \multicolumn{4}{g} {\footnotesize{7 class-based models}}&
        \multicolumn{3}{g}{}&
        \multicolumn{4}{g|} {\footnotesize{8 class-based models}}&
        \multicolumn{1}{g|}{\footnotesize{Dist}}\\
        
        \cline{3-6}\cline{10-13}\multicolumn{1}{|g}{}&
        \multicolumn{1}{|g|}{\footnotesize{ance}}&
        \multicolumn{1}{g}{acc}&\multicolumn{1}{g}{sens}&
        \multicolumn{1}{g}{spec}&\multicolumn{1}{g}{mcc}&
        \multicolumn{3}{g}{}&
        \multicolumn{1}{g}{acc}&\multicolumn{1}{g}{sens}&
        \multicolumn{1}{g}{spec}&\multicolumn{1}{g|}{mcc}&\multicolumn{1}{g|}{\footnotesize{ance}}\\

        \hline

        \multicolumn{1}{|l|}{\multirow{6}{*}{\footnotesize{Prediction-based}}}
        
        & 0.70 & 76.88 & 19.10 & 86.51 & 0.05 &    s&&s                & 84.58 & 38.33 & 91.19 & 0.29 & 0.45 \\
        & 0.39 & 83.66 & 42.86 & 90.47 & 0.33 &    d&\small{Micro}&d   & 84.79 & 39.16 & 91.31 & 0.30 & 0.44 \\
        & \mrkm{0.36}{84.43}{45.51}{90.91}{0.36}&    sh&&sh              & \mrkm{88.71}{54.85}{93.54}{0.48}{0.32} \\

        
        \cline{2-5}\cline{7-9}\cline{11-14}
        
        & 0.81 & 76.88 & 6.69 & 81.70 & 0.01 &    s&&s                & 84.58 & 76.82 & 86.79 & 0.03 & 0.81 \\
        & 0.39 & 83.67 & 42.85 & 90.47 & 0.32 &    d&\small{Macro}&d   & 84.79 & 39.16 & 91.31 & 0.29 & 0.44 \\
        & 0.39 & 85.27 & 43.31 & 90.33 & 0.35 &    sh&&sh              & 88.71 & 35.32 & 91.94 & 0.29 & 0.48 \\
        
        \hline\hline
        
        \multicolumn{14}{|c|} {\footnotesize{
            S, D and SH are sorted, down-sampled and shuffled instances of the main dataset.
        }}\\
        \multicolumn{14}{|c|} {\footnotesize{
            Acc: Accuracy, Sens: Sensitivity, Spec: Specificity, Mcc: Matthews correlation coefficient
        }}\\

        \hline
        
       

    \end{tabular}
    \captionsetup{font=small,width=14cm}
    \caption{The average sensitivity, specificity, accuracy, and MCC values 
    for scikit-learn prediction-based models for amino acid composition (AAC).}
    \label{tab:table1}
    
\end{table}

    \subsubsection{One vs All model}
    Considering the differences in between our results from the basic Scikit-Learn model and the ones of 
    the paper on AAC feature, we tried a One vs All model to improve the results. In this model, we coded all 
    the phases without making any call on functions from different libraries (except for the classification algorithm) 
    to gain more insight on the other problem-related parameters that could affect the model performance.\\

    \myparagraph{Construction of the main dataset}
    
    To observe In this Model, we experimented on the datasets with seven (transporters) and eight 
    (transporters and non-transporters) classes of sequences. The 7 class-based dataset contains 
    amino acid composition vectors for 780 transporter proteins classified into seven substrate-specific 
    classes(amino acid, anion, cation, electron, others, protein and sugar). The 8-class based dataset 
    contains 1380 amino acid composition vectors for 600 non-transporter proteins and 780 transporter 
    proteins mentioned above.\\

    We then generated the normal, shuffled and downsampled instances of the main dataset for each of 
    those sets mentioned above to observe how changes in the order and number of the vectors can affect 
    the results in an imbalanced environment.\\

    For the normal instance of the dataset, we classified all sequence under seven 
    (for the 7 class-based transporter set) or eight (for 8 class-based transporter and non-transporter set) 
    categories of correspondent classes. We then applied 5-fold cross-validation to each category and 
    created 5 folds with its correspondent train and test sets for each fold. For the final folds 
    (i.e. final dataset's first fold), we concatenated the related fold from all the classes available in the set 
    (i.e. first fold from the amino acid, anion, cation, etc.). Considering the imbalanced nature of our dataset, 
    through each fold of the final dataset, this approach provides an equal portion of each class in both the train 
    (4/5th of the class sequences) and the test (1/5th of the class sequences) sets.\\

    For the shuffled set, we first, shuffled the main dataset using a random seed of 2. 
    We then applied 5-fold cross-validation and created the correspondent train and test sets.\\

    The down-sampled instance of the main dataset contains 60 sequences from each class 
    (number of records in our anion class with the least amount of sequences) being taken 
    randomly from that specific class. We then classified them under seven (for the 7 class-based sets) 
    or eight (for 8 class-based sets) categories, applied 5-fold cross-validation to each one of those classes and 
    finally concatenated the correspondent folds from each category in a random order (using the random seed mentioned above) 
    to create a final balanced train and test sets for each fold. The final number of records are 420 for 
    7-class based set and 480 for the 8 class-based sets.\\


    For each one of those 6 datasets mentioned above (normal, shuffled and down-sampled 
    for 7 and 8 class-based sets), and for each fold, we transformed the set into "One against All" format. 
    In this format, in each fold, from each set (train or test), we produced seven (for the 7 class-based sets) 
    or eight (for 8 class-based sets) sets in a way that, for each class (i.e. anion transporters), 
    we labelled the records from that class (i.e. anion) with "+1" and the rest of the classes with "-1". 
    So, for example, for fold one of 7-class based dataset, we generated 7 new sets through which for amino acid class, 
    there are 70 records (amino acid transporters) with "+1" label and 710 other records with "-1" label. 
    The same thing is done for the rest of the classes in the same fold. A sample generated set is available at 
    the paper’s GitHub repository. \\


    \myparagraph{Classification}

    \myparagraph{\footnotesize{Choice of classification algorithm}}

        In this paper, for the classification algorithm, we both experimented with SVMLight software, 
        and the Scikit-Learn's "SVC" classifiers in our models. \\

    \myparagraph{\footnotesize{Gamma and Cost}}

        Considering the imbalanced nature of our dataset, assigning one single pair of Gamma and Cost values to 
        all classes, would lead to less true positives for some involved classes in each fold. So, to observe the 
        impact on the final performance metrics, we once tried classification with 
        the best pair for all classes (being chosen through Grid Search) and the other time we tried classification with 
        assigning different Gamma and Cost value pairs to each class which was the best one for that specific class.
        The assigned value pairs are the result of applying Grid Search to that specific class instead of the whole dataset).\\ 

        So, for each dataset (7 or 8 class-based sets), we tried 6 different 
        scenarios (i.e. shuffled 7-class set with one Gamma and Cost values for all classes versus the same set with values 
        different for each class.)\\

    \myparagraph{Model performance metrics}
        
        In this study, sensitivity, specificity, accuracy and Matthew's correlation coefficient (MCC) is used to measure the 
        prediction performance. TP, FP, TN and FN are true positives, false positives, 
        true negatives and false negatives, respectively. \\

        Sensitivity (also termed as Recall) is a measure of the proportion of actual positive cases that is 
        predicted as positive (or true positive). In this study, sensitivity is the percentage of correctly predicted transporters 
        and is calculated as:
        \begin{equation}
            \text{Sensitivity} = \frac {\text{TP}}{\text{TP + FN}} * 100
        \end{equation}

        Specificity is defined as the proportion of actual negatives which is predicted as the negative (or true negative). It is  
        the percentage of non-transporters that were correctly predicted as non-transport proteins and is computed as:
        \begin{equation}
            \text{Specificity} = \frac {\text{TN}}{\text{TN + FP}} * 100
        \end{equation}

        Accuracy is the overall percentage of the correctly predicted classes (true positives and true negatives) to all the 
        predictions. In this study, it's the overall percentage of transporter and non-transporter proteins 
        being predicted correctly. It's calculated as:
        \begin{equation}
            \text{Accuracy} = \frac {\text{TP + TN}}{\text{TP + FP + TN + FN}} * 100
        \end{equation}

        Introduced by Brian W. Matthews in 1975, in machine learning, the Matthews correlation coefficient (MCC) 
        is used as a measure of the quality of binary classifications. The coefficient takes into account true 
        and false positives and negatives. It is generally regarded as a balanced measure which can be used to 
        the imbalanced classification problems where the distribution of examples across the classes is not equal.\cite{mcc2017optimal}
        The MCC returns a value between -1 and +1. A coefficient of +1 represents a perfect prediction, 0 no better than 
        random prediction and -1 indicates total disagreement between prediction and observation. The MCC is in 
        essence a correlation coefficient between the observed and predicted binary classifications and is computed as:
        
        \begin{equation}
            \text{MCC} = \frac {\text{(TP * TN) - (FN * FP)}}{\sqrt{(TP + FN) * (TN + FP) * (TP + FP) * (TN + FN)}}
        \end{equation}

    \myparagraph{Evaluation}

        We considered three different models for evaluating the results after classification. For example in case of a 
        shuffled 7 class-based set, after classification, there would be 7 different result sets in each fold. 
        These are generated from applying seven classifiers to each one of those involved classes.\\ 

        The first approach is to calculate the performance metrics (accuracy, sensitivity, specificity, MCC) for 
        each one of those classes in a fold and then produce the metrics for a fold by averaging in between the 
        correspondent values from each class. The final metrics for a model could be calculated by averaging in 
        between the results of all five folds. Through Table \ref{tab:table4} and \ref{tab:table5}, 
        we referred to this model as “class-based” evaluation.\\

        We obtained different results from the model above. To balance out the specificity and sensitivity values, 
        we tried another model through which we applied a single threshold value to all available classification results 
        in each fold. We then calculated the performance metrics for each class. Averaging the correspondent metrics 
        values from all classes within a fold would produce the metrics for that fold like the model above. 
        The final results would be then calculated by averaging in between the results from all five folds. 
        We referred to this process as “threshold-based” evaluation in Table \ref{tab:table4} and \ref{tab:table5}.\\

        We achieved good results from the thresholding model, but the two models mentioned above could produce more 
        than one right label for a sequence. So, we then designed another experiment through which we voted in 
        between all those seven classification results and produced one result for each fold. We then calculated 
        accuracy, specificity, sensitivity and MCC for that fold and averaged in between all five folds 
        to produce the final performance metrics for that model. We referred to this model as “vote-based” 
        evaluation through Table \ref{tab:table4} and \ref{tab:table5}.\\

    \myparagraph{\footnotesize{Micro averaging vs Macro averaging}}

        There are two different approaches for calculating the performance metrics for a fold (or a model if there are no folds). 
        The process starts with computing the confusion matrix elements (true positive, true negative, false positive 
        and false negative) for all the involved classes. We can then either calculate the metrics for each class 
        independently (i.e. accuracy) and then average in between all those metrics for the final results for a fold 
        (Macro averaging) or we can sum all the TPs, TNs, FPs and FNs obtained from each class and then calculate the metrics 
        for that fold using those values (Micro averaging).\\
    
        Micro or macro-averaging (for any metric) computes slightly different things, and thus their interpretation differs.  
        A macro-average computes the metric independently for each class and then takes the average 
        (hence treating all classes equally), whereas a micro-average aggregates the contributions of all classes 
        to compute the average metric. In this study and for each model, we tried both micro and macro averaging approaches 
        and reported the results in table \ref{tab:table4} and \ref{tab:table5}.\\


    \myparagraph{Results}

        Table \ref{tab:table4} and \ref{tab:table5} show the average accuracy, sensitivity, specificity and 
        Mathews correlation coefficient values for all the 144 models being built on seven and eight class-based datasets 
        through our one vs all approach. Table \ref{tab:table4} corresponds to the results from the models being built on 
        seven transporters class dataset (780 sequences) and \ref{tab:table5} reports on the performance metrics of 
        the models using seven transporters and one non-transporter class (1380 sequences) as their main dataset.\\

        Compared to original results, some models produced higher values for accuracy and specificity, a lower value for 
        sensitivity and a MCC value close to the reported one. Some others produced close values for accuracy, sensitivity 
        and specificity but lower value for the MCC. So, we introduced a distance value as a measure for making a comparison 
        in-between all the 144 models. The distance value for each model in Table \ref{tab:table4} and \ref{tab:table5}, 
        is the distance in between the results of that specific model and the originally reported values.\\
        
        Overall, the 8 class-based models produce higher values for all 4 involved performance metrics compared to the same models 
        being built on the dataset with 7 transporter classes due to having 600 more sequences in the main dataset which means more 
        information through the training phase. But higher values also means a greater distance from the original results. 
        With 7 class models we managed to obtain close balanced values for accuracy, sensitivity and specificity, 
        but the MCC was always less than the original one. On the 8 class-based dataset models, we achieved higher values for 
        all those four metrics. So, adding a class provides better results and a closer MCC value which also increases the distance 
        in between the original and the reproduced results.\\
        
        Regarding the different instances of a dataset (normal, shuffled and downsampled version of an i.e. seven class-based set), 
        the results are very close for normal and shuffled instances. That is due to the fact that although the number of sequences 
        (form each class) available in the train and test sets could be different from a normal set to a shuffled one, 
        but the difference is not that much. Normal and shuffled instances are both imbalanced sets but normal one provides 
        an equal portion of each class in both the train and the test sets while through shuffled one, although we have sequences 
        from all classes in the train and test sets of each fold, but it would be a different portion from fold to fold. 
        The downsampled instance is the balanced version with much less sequences in it (420 records compared to 780 for seven 
        class-based set and 480 vectors compared to 1380 for 8 class-based sets). So, due to the availability of less but balanced 
        information, we obtained more balanced results but the metrics values are less than those of the shuffled and normal ones.\\ 
        
        Considering the imbalanced nature of shuffled and normal instances, micro and macro-averaging produce different results 
        for these two sets. But it does not have a noticeable impact on the final results in case of downsampled instance. 
        Micro averaging provides higher values for sensitivity and MCC metrics which leads to more balanced result 
        with less distance to the original ones.\\
        
        Assigning different Gamma and Cost value pairs to each class, produces better results compared to assigning 
        one single value pair to all classes. That is due to the fact that assigning different values to each class that is 
        accustomed to the distribution of values in that specific class, produces more true positives for some classes after 
        classification which leads to a higher true positive value for that fold. But overall the results are close in both cases. \\
        
        Compared to SVC function from scikit-learn, SVMLight performs better on class-based model. In threshold-based model, 
        Scikit-learn provides results closer to the original ones while SVMLight provides higher MCC value and they both produce 
        close results in vote-based model.\\
        
        Considering the presence of multiple classes in the dataset and consequently multiple classifiers in our 
        one vs all model,the class-based model could be used where there could more than one right answer (label) 
        to the problem (multi-label classification). If that is the case, then in an imbalanced problem like ours, 
        applying a threshold to the classification results (threshold-based model) can provide a balanced results 
        for the problem. The vote-based model is the case for multi-class classification problem where there could 
        be one right label for a classification problem.\\
        
        Overall, the closest results are achieved with micro-averaging the results on “threshold-based” model with either 
        normal or shuffled instance of the main dataset. Using seven transporters class dataset, the MCC value and the distance 
        between reproduced and original results are less than the ones of the same model being built on eight class-based dataset, 
        but the latter shows better performance and provides higher values for performance metrics. 
        SVMLight and SVC function from Scikit-Learn both produce the highest values with micro-averaging the results 
        on “vote-based” models with either normal or shuffled instance of the dataset.\\
    
    % 
\begin{table}[ht]
    \centering
    \begin{tabular}{|L |V V V V g g g V V V V |}
        \hline
        \multicolumn{12}{|g|}{7-class based model for AAC}\\
        \hline
        \multicolumn{1}{|l|}{\multirow{2}{*}{\footnotesize{Original Results}}}
        &
        \multicolumn{2}{g}{Accuracy} & \multicolumn{3}{g}{Sensitivity} &
        \multicolumn{3}{g}{Specificity} & \multicolumn{3}{g|}{MCC}\\
        \cline{2-12}&
        \multicolumn{2}{C}{73.74} & \multicolumn{3}{C}{74.65} &
        \multicolumn{3}{C}{73.22} & \multicolumn{3}{C|}{0.46} \\
        \hline\hline
        
        \multicolumn{12}{|g|} {SVM Light}\\
        \hline\hline
        \cline{1-12}\multicolumn{1}{|g}{}&
        \multicolumn{4}{g} {Gamma-Cost different for each class}&
        \multicolumn{3}{g}{}&
        \multicolumn{4}{g|} {same Gamma-Cost for all classes}\\
        \cline{2-5}\cline{9-12}\multicolumn{1}{|g}{}&
        \multicolumn{1}{g}{accuracy}&\multicolumn{1}{g}{sensitivity}&
        \multicolumn{1}{g}{specificity}&\multicolumn{1}{g}{mcc}&
        \multicolumn{3}{g}{}&
        \multicolumn{1}{g}{accuracy}&\multicolumn{1}{g}{sensitivity}&
        \multicolumn{1}{g}{specificity}&\multicolumn{1}{g|}{mcc}\\
        \hline

        \multicolumn{1}{|l|}{\multirow{6}{*}{\footnotesize{Before Voting}}}
        
        & 82.28 & 56.53 & 86.58 & 0.37 &    n&&n                & 80.73 & 60.00 & 84.18 & 0.37 \\
        & 84.07 & 39.28 & 91.55 & 0.32 &    d&\small{Micro}&d   & 81.90 & 48.33 & 87.5 & 0.32 \\
        & 81.74 & 54.74 & 86.24 & 0.36 &    sh&&sh              & 80.42 & 59.61 & 83.88 & 0.36 \\
        
        \cline{3-4}\cline{6-8}\cline{10-11}
        
        & 82.28 & 43.76 & 84.31 & 0.31 &    n&&n                & 80.73 & 46.64 & 81.69 & 0.30 \\
        & 84.08 & 39.28 & 91.54 & 0.32 &    d&\small{Macro}&d   & 81.90 & 48.33 & 87.49 & 0.33 \\
        & 81.73 & 43.27 & 84.06 & 0.28 &    sh&&sh              & 80.42 & 46.96 & 81.49 & 0.29 \\
        
        \hline
        \multicolumn{1}{|l|}{\multirow{6}{*}{\footnotesize{Thresholding}}}

        & 74.04 & 74.74 & 73.93 & 0.36 &    n&&n                & 73.07 & 76.66 & 72.47 & 0.36 \\
        & 72.89 & 66.66 & 73.92 & 0.30 &    d&\small{Micro}&d   & 69.96 & 71.19 & 69.76 & 0.29 \\
        & 73.88 & 72.18 & 74.16 & 0.34 &    sh&&sh              & 72.91 & 75.38 & 72.50 & 0.35 \\
        
        \cline{3-4}\cline{6-8}\cline{10-11}

        & 74.04 & 63.58 & 70.80 & 0.28 &    n&&n                & 73.07 & 65.34 & 69.17 & 0.27 \\
        & 72.89 & 66.66 & 73.92 & 0.30 &    d&\small{Macro}&d   & 69.96 & 71.19 & 69.76 & 0.30 \\
        & 73.88 & 62.26 & 71.01 & 0.27 &    sh&&sh              & 72.91 & 64.46 & 69.24 & 0.26 \\
        
        \hline
        \multicolumn{1}{|l|}{\multirow{6}{*}{\footnotesize{After Voting}}}

        & 84.87 & 47.05 & 91.17 & 0.38 &    n&&n                & 84.54 & 45.89 & 90.98 & 0.36 \\
        & 84.35 & 45.24 & 90.87 & 0.36 &    d&\small{Micro}&d   & 83.87 & 43.57 & 90.59 & 0.34 \\
        & 84.32 & 45.12 & 90.85 & 0.35 &    sh&&sh              & 84.21 & 44.74 & 90.79 & 0.35 \\
        
        \cline{3-4}\cline{6-8}\cline{10-11}

        & 84.87 & 34.05 & 89.58 & 0.28 &    n&&n                & 84.54 & 33.59 & 89.48 & 0.27 \\
        & 84.35 & 45.23 & 90.87 & 0.34 &    d&\small{Macro}&d   & 83.87 & 43.57 & 90.59 & 0.33 \\
        & 84.32 & 33.52 & 89.14 & 0.27 &    sh&&sh              & 84.21 & 33.47 & 89.15 & 0.27 \\
        
        \hline\hline
        
        \multicolumn{12}{|g|} {Scikit Learn}\\
        \hline\hline

        \multicolumn{1}{|l|}{\multirow{6}{*}{\footnotesize{Before Voting}}}
        
        & 87.42 & 34.10 & 96.30 & 0.39 &    n&&n                & 80.25 & 30.90 & 88.48 & 0.19 \\
        & 87.07 & 24.28 & 97.53 & 0.33 &    d&\small{Micro}&d   & 86.62 & 25.00 & 96.90 & 0.31\\
        & 86.75 & 31.92 & 95.90 & 0.36 &    sh&&sh              & 87.01 & 30.89 & 96.36 & 0.36\\
        
        \cline{3-4}\cline{6-8}\cline{10-11}
        
        & 87.41 & 28.70 & 95.77 & 0.34 &    n&&n                 & 87.20 & 24.18 & 96.06 & 0.28 \\
        & 87.07 & 24.28 & 97.53 & 0.30 &    d&\small{Macro}&d   & 86.63 & 25.00 & 96.90 & 0.28 \\
        & 86.75 & 26.20 & 95.36 & 0.29 &    sh&&sh              & 87.014 & 24.19 & 95.84 & 0.27 \\
        
        \hline
        \multicolumn{1}{|l|}{\multirow{6}{*}{\footnotesize{Thresholding}}}

        & 74.52 & 35.26 & 81.07 & 0.14 &    n&&n                & 75.07 & 31.54 & 82.33 & 0.12 \\
        & 80.13 & 24.52 & 89.40 & 0.14 &    d&\small{Micro}&d   & 79.25 & 25.00 & 88.29 & 0.13\\
        & 73.07 & 32.81 & 79.78 & 0.10 &    sh&&sh               & 73.90 & 31.66 & 80.94 & 0.11\\
        
        \cline{3-4}\cline{6-8}\cline{10-11}

        & 74.52 & 30.74 & 81.54 & 0.13 &    n&&n                & 75.07 & 25.34 & 82.74 & 0.09\\
        & 80.13 & 24.52 & 89.40 & 0.17 &    d&\small{Macro}&d   & 79.24 & 25.00 & 88.29 & 0.15\\
        & 73.07 & 27.74 & 80.28 & 0.09 &    sh&&sh              & 73.90 & 25.36 & 81.40 & 0.08 \\
        
        \hline
        \multicolumn{1}{|l|}{\multirow{6}{*}{\footnotesize{After Voting}}}

        & 81.28 & 34.49 & 89.08 & 0.24 &    n&&n                & 80.25 & 30.90 & 88.49 & 0.19\\
        & 78.84 & 25.95 & 87.65 & 0.13 &    d&\small{Micro}&d   & 79.18 & 27.14 & 87.86 & 0.15\\
        & 80.66 & 32.308 & 88.72 & 0.21&    sh&&sh              & 80.51 & 31.79 & 88.63 & 0.20\\
        
        \cline{3-4}\cline{6-8}\cline{10-11}

        & 81.28 & 29.66 & 89.07 & 0.19 &    n&&n                & 80.26 & 24.75 & 88.53 & 0.13 \\
        & 78.84 & 25.95 & 87.65 & 0.14&     d&\small{Macro}&d   & 79.18 & 27.14 & 87.85 & 0.14\\
        & 80.65 & 27.29 & 88.71 & 0.15 &    sh&&sh              & 80.51 & 25.97 & 88.65 & 0.14 \\
        \hline
        
       

    \end{tabular}
    \captionsetup{font=small,width=12cm}
    \caption{The average sensitivity, specificity, accuracy, and MCC from all the 
    models being built on 7 class-based dataset. n, d and sh are normal, 
    downsampled and shuffled instances of the main dataset.}
    \label{tab:table4}
    
\end{table}
    % \begin{table}[ht]
    \centering
    \begin{tabular}{|L | V |V V V V g g g V V V V | V |}
        \hline
        \multicolumn{14}{|g|}{8-class based model for AAC}\\
        \hline
        \multicolumn{1}{|l|}{\multirow{2}{*}{\footnotesize{Original Results}}}
        &
        \multicolumn{3}{g}{Accuracy} & \multicolumn{3}{g}{Sensitivity} &
        \multicolumn{3}{g}{Specificity} & \multicolumn{4}{g|}{MCC}\\
        \cline{2-14}&
        \multicolumn{3}{C}{73.74} & \multicolumn{3}{C}{74.65} &
        \multicolumn{3}{C}{73.22} & \multicolumn{4}{C|}{0.46} \\
        \hline\hline
        
        \multicolumn{14}{|g|} {SVM Light}\\
        \hline\hline
        
        \cline{1-12}\multicolumn{1}{|g}{}&
        
        \multicolumn{1}{|g|}{\footnotesize{Dist}}&
        \multicolumn{4}{g} {Gamma-Cost different for each class}&
        \multicolumn{3}{g}{}&
        \multicolumn{4}{g|} {same Gamma-Cost for all classes}&
        \multicolumn{1}{g|}{\footnotesize{Dist}}\\
        
        \cline{3-6}\cline{10-13}\multicolumn{1}{|g}{}&
        \multicolumn{1}{|g|}{\footnotesize{ance}}&
        \multicolumn{1}{g}{acc}&\multicolumn{1}{g}{sens}&
        \multicolumn{1}{g}{spec}&\multicolumn{1}{g}{mcc}&
        \multicolumn{3}{g}{}&
        \multicolumn{1}{g}{acc}&\multicolumn{1}{g}{sens}&
        \multicolumn{1}{g}{spec}&\multicolumn{1}{g|}{mcc}&\multicolumn{1}{g|}{\footnotesize{ance}}\\

        \hline

        \multicolumn{1}{|l|}{\multirow{6}{*}{\footnotesize{Class-based}}}
        
        & 0.24 & 87.33 & 64.92 & 90.53 & \mrko{0.49} &    b&&b               & 86.22 & 63.40 & 89.48 & 0.46 & 0.23 \\
        & 0.49 & 85.86 & 34.16 & 93.24 & 0.30 &    d&\small{Micro}&d   & 85.02 & 40.83 & 91.34 & 0.32 & 0.42 \\
        & 0.24 & 87.02 & 63.98 & 90.31 & 0.48 &    s&&s                & 86.24 & 62.46 & 89.63 & 0.46 & 0.23 \\
        
        \cline{2-5}\cline{7-9}\cline{11-14}
        
        & 0.43 & 87.33 & 39.91 & 88.02 & 0.30 &    b&&b               & 86.22 & 34.78 & 86.71 & 0.25 & 0.48 \\
        & 0.49 & 85.85 & 34.16 & 93.24 & 0.28 &    d&\small{Macro}&d   & 85.02 & 40.83 & 91.34 & 0.30 & 0.42 \\
        & 0.44 & 87.02 & 38.81 & 87.71 & 0.28 &    s&&s                & 86.24 & 33.57 & 86.89 & 0.23 & 0.50 \\
        
        \hline
        \multicolumn{1}{|l|}{\multirow{6}{*}{\footnotesize{Threshold-based}}}

        & \mrkm{0.10}{79.98}{79.78}{80.01}{0.44} &    b&&b               & \mrkm{79.74}{77.97}{80.00}{0.43}{0.10} \\
        & 0.19 & 68.93 & 71.87 & 68.51 & 0.27 &     d&\small{Micro}&d   & 68.07 & 69.79 & 67.82 & 0.25 & 0.22 \\
        & \mrkm{0.09}{77.71}{79.93}{77.40}{0.41} &     s&&s                & \mrkm{78.31}{78.33}{78.31}{0.41}{0.09} \\
        
        \cline{2-5}\cline{7-9}\cline{11-14}

        & 0.21 & 79.59 & 57.34 & 76.24 & 0.28 &    b&&b               & 79.59 & 57.34 & 76.24 & 0.28 & 0.25 \\
        & 0.19 & 68.93 & 71.87 & 68.51 & 0.27 &     d&\small{Macro}&d   & 68.07 & 69.79 & 67.82 & 0.25 & 0.22 \\
        & 0.21 & 77.71 & 53.55 & 74.14 & 0.27 &     s&&s                & 78.31 & 58.42 & 74.72 & 0.26. & 0.26 \\
        
        \hline
        \multicolumn{1}{|l|}{\multirow{6}{*}{\footnotesize{Vote-based}}}

        & 0.31 & 88.91 & 55.65 & 93.66 & \mrko{0.49} &    b&&b               & 88.51 & 54.05 & 93.43 & 0.47 & 0.32 \\
        & 0.43 & 84.84 & 39.37 & 91.33 & 0.30 &    d&\small{Micro}&d   & 84.79 & 39.16 & 91.30 & 0.30 & 0.44 \\
        & 0.32 & 88.44 & 53.76 & 93.39 & 0.47 &    s&&s                & 88.33 & 53.33 & 93.33 & 0.46 & 0.32 \\
        
        \cline{2-5}\cline{7-9}\cline{11-14}

        & 0.54 & 88.91 & 29.87 & 91.65 & 0.26 &    b&&b               & 88.51 & 25.93 & 91.35 & 0.22 & 0.58 \\
        & 0.44 & 84.84 & 39.37 & 91.34 & 0.29 &    d&\small{Macro}&d   & 84.79 & 39.16 & 91.31 & 0.28 & 0.44 \\
        & 0.57 & 88.44 & 27.85 & 91.28 & 0.23 &    s&&s                & 88.33 & 24.93 & 91.22 & 0.20 & 0.60 \\
        
        \hline
        \hline
        
        \multicolumn{14}{|g|} {Scikit Learn}\\
        \hline
        \hline

        \multicolumn{1}{|l|}{\multirow{6}{*}{\footnotesize{Class-based}}}
        
        & 0.48 & 90.22 & 36.33 & 97.84 & 0.46  &    b&&b               & 90.11 & 35.22 & 97.92 & 0.45 & 0.49 \\
        & 0.72 & 88.10 & 12.18 & 98.86 & 0.23 &    d&\small{Micro}&d   & 87.75 & 9.63 & 98.86 & 0.19 & 0.76  \\
        & 0.48 & 90.10 & 36.52 & 97.68 & 0.45 &    s&&s                & 89.89 & 33.28 & 97.96 & 0.43 & 0.50 \\
        
        \cline{2-5}\cline{7-9}\cline{11-14}
        
        & 0.68 & 90.19 & 17.30 & 96.85 & 0.22 &    b&&b               & 90.10 & 15.26 & 96.98 & 0.17 & 0.71 \\
        & 0.74 & 88.10 & 12.21 & 98.86 & 0.18 &    d&\small{Macro}&d   & 87.75 & 9.71 & 8.86 & 0.13 & 0.78  \\
        & 0.68 & 90.07 & 17.17 & 96.68 & 0.20 &    s&&s                & 89.88 & 12.90 & 97.02 & 0.15 & 0.74 \\
        
        \hline
        \multicolumn{1}{|l|}{\multirow{6}{*}{\footnotesize{Threshold-based}}}

        & \mrkm{0.08}{77.08}{80.0}{76.66}{0.40}  &     b&&b               & \mrkm{77.70}{78.18}{77.63}{0.40}{0.08} \\
        & 0.22 & 67.91 & 69.16 & 67.73 & 0.25 &     d&\small{Micro}&d   & 68.75 & 67.28 & 68.96 & 0.25 & 0.23  \\
        & \mrkm{0.08}{76.66}{78.47}{76.40}{0.39} &     s&&s                & 74.36 & 79.41 & 73.64 & 0.37 & 0.09 \\
        
        \cline{2-5}\cline{7-9}\cline{11-14}

        & 0.24 & 77.08 & 59.13 & 73.10 & 0.27 &     b&&b               & 77.70 & 54.72 & 74.14 & 0.25 & 0.28 \\
        & 0.21 & 67.91 & 69.16 & 67.73 & 0.26 &     d&\small{Macro}&d   & 68.75 & 67.29 & 68.95 & 0.26 & 0.22  \\
        & 0.27 & 76.66 & 56.55 & 72.78 & 0.25 &     s&&s                & 74.36 & 56.27 & 0.07 & 0.22 & 0.30 \\
        
        \hline
        \multicolumn{1}{|l|}{\multirow{6}{*}{\footnotesize{Vote-based}}}

        & 0.31 & 89.33 & 57.31 & 93.90 & \mrko{0.51} &     b&&b               & 88.73 & 54.92 & 93.56 & 0.48 & 0.32 \\
        & 0.42 & 85.10 & 40.41 & 91.49 & 0.31 &     d&\small{Micro}&d   & 85.31 & 41.25 & 91.60 & 0.32 & 0.41  \\
        & 0.31 & 89.09 & 56.37 & 93.76 & 0.50 &     s&&s                & 88.15 & 52.60 & 93.22 & 0.45 & 0.33 \\
        
        \cline{2-5}\cline{7-9}\cline{11-14}

        & 0.47 & 89.32 & 36.12 & 91.94 & 0.34 &     b&&b               & 88.73 & 31.39 & 91.51 & 0.29 & 0.52 \\
        & 0.43 & 85.10 & 40.41 & 91.48 & 0.30 &     d&\small{Macro}&d   & 85.31 & 41.25 & 91.60 & 0.31 & 0.42  \\
        & 0.49 & 89.09 & 33.78 & 91.73 & 0.30 &     s&&s                & 88.15 & 28.05 & 91.14 & 0.23 & 0.56 \\
        \hline\hline
        
        \multicolumn{14}{|c|} {\footnotesize{
            B, D and S are balanced, down-sampled and shuffled instances of the main dataset.
        }}\\
        \multicolumn{14}{|c|} {\footnotesize{
            Acc: Accuracy, Sens: Sensitivity, Spec: Specificity, Mcc: Matthews correlation coefficient
        }}\\

        \hline
        
       

    \end{tabular}
    \captionsetup{font=small,width=14cm}
    \caption{The average sensitivity, specificity, accuracy, and MCC  for 8 class-based models.}
    \label{tab:table5}
    
\end{table}




% =======================================================================================================================
% =======================================================================================================================