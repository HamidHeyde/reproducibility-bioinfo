\section{Materials and Methods}
\label{sec:materials}
The materials and methods described in this section present a replication of the study performed
by~\cite{mishra2014prediction}. In cases where insufficient details were provided for replication,
these decision points were noted and several sensible options selected and compared. All software 
developed to curate the dataset and perform the experiment, including a Jupyter notebook to reproduce our key results, can be found publicly available on
our GitHub repository: \url{https://github.com/big-data-lab-team/reproducibility-bioinfo/}.

\subsection{Dataset}
\label{sec:dataset}

The SwissProt UniProt database \sout{consisting of $10,780$ proteins} with rich sequence and substrate
annotations~\cite{boeckmann2003swiss} was subsampled to include $780$ membrane transporter proteins and $660$
non-transporter proteins. The transport proteins were divided into $7$ substrate-specific classes ($70$ amino acid
transporters, $60$ anion transporters, $260$ cation transporters, $60$ electron transporters, $60$ sugar
transporters, $70$ protein/mRNA transporters, and $200$ other transporters). With the addition of 600 non-transporter
proteins, the total dataset contained $1,380$ protein sequences. \GK{Hamid: state explicitly the discrepancy between
the 780 transporters we are using versus the 900 included in the original paper. This total appears to correspond to
the ''main'' section of the dataset described in the original paper, which suggests we threw out the validation set?
We should be clear about this}

\Hamid{About the SwissProt, the initial $10,780$ sequences in the dataset, how they got to the final 
dataset from there, and the subset we used for our experiment :: i wasn't sure how to mix it in with the text.
[i can also add a table showing the number of proteins in each one]}
{
The dataset we used for this study is the same as the one being used in the initial work which is 
available at \url{http://bioinfo.noble.org/TrSSP}. The authors in~\cite{mishra2014prediction} were initially collected 
10,780 transporter, carrier, and channel proteins that were well characterized at the protein level and had clear substrate 
annotations from the SwissProt(release 2013-03)~\cite{boeckmann2003swiss}. They then removed the transporters 
with more than two substrate specificities, sequences with biological function annotations based solely on sequence 
similarity, and sequences with greater than 70\% similarity. The final dataset contains 660 non-transporters and 
900 transporters being divided into following 7 substrate classes: amino acid, anion, cation, electron, protein/mRNA, 
sugar, and other.\\
These $1,560$ sequences were then divided into two datasets: 1) Main dataset for the total of $1,380$ sequences 
and 2) Independent Dataset which contains $180$ sequences. The work in~\cite{mishra2014prediction} reports on results of 
all the developed models on the main dataset and the results from the best model (AAIndex+PSSM) on the independent set. 
Through this study, we used the main dataset and experimented on all the models being developed on that set. 
The main dataset contains $600$ non-transporter proteins and $780$ transporter proteins being divided into $7$ 
substrate-specific classes ($70$ amino acid transporters, $60$ anion transporters, $260$ cation transporters, 
$60$ electron transporters, $60$ sugar transporters, $70$ protein/mRNA transporters, and $200$ other transporters). 
The sequences are also available in our GitHub repository: \url{https://github.com/big-data-lab-team/reproducibility-bioinfo/}.
}\\

Features were computed for each protein, including: Amino Acid Composition (AAC), Dipeptide Composition (DPC),
Physico-Chemical Composition (PHC), Biochemical Composition (AAindex) and Position-specific scoring matrix (PSSM)
profile. Each feature was computed identically to the methods described by~\cite{mishra2014prediction} and are briefly
summarized below:

\begin{itemize}
\item \textbf{Amino Acid Composition (AAC)}: a feature vector of $20$ values ranging from $0$ -- $100$ indicating the
percentage of all standard amino acids present within a protein, as defined by~\cite{gromiha2010protein}. Also known
as Monopeptide Composition (MPC).
\item \textbf{Dipeptide Composition (DPC)}: a feature vector of $400$ values ranging from $0$ -- $100$ indicating the
percentage of all possible ordered amino acid pairs present within a protein, as defined by~\cite{gromiha2010protein}.
\item \textbf{Physico-Chemical Composition (PHC)}: a feature vector of $11$ values corresponding
to percentage composition of physico-chemical residue classes, including: Aliphatic, Neutral, Aromatic, Hydrophobic, Charged, Positively charged,
Negatively charged, Polar, Small, Large, and Tiny. 
\item \textbf{Biochemical Composition (AAindex)}: a feature vector of $49$ physical, chemical, energetic, and
conformational amino acid properties which have been averaged across all amino acids present within the protein.
\item \textbf{Position-specific scoring matrix (PSSM) profile}: a feature vector of $400$ sequence
likelihoods aggregated across min-max scaled probabilities for each ordered amino acid pair within
proteins in the SwissProt database.
\end{itemize}

\TG{Add a sentence to explain that no difference was observed between the features that we computed and the original ones.}
\Hamid{
Regarding the comment above 
:: I added the following sentence but, the reality is that i couldn't find the computed 
features from the initial study anywhere. So, based on what i could find online from other people i computed the first 3. 
and for the last 2, i just talked to Munira about it and that was one of my discussion points. 
:: but i'm not insisting on it. just wanted to let you guys know. if it's better this way, i'll roll along.
}
{The computed features were compared against the ones of the initial work~\cite{mishra2014prediction} and no difference was 
observed in that regard.}
% =====================================================================================================================
% =====================================================================================================================
\subsection{Model Flexibility}
\label{sec:modelflex}
Though the majority of dataset and model specifications were clearly specified by~\cite{mishra2014prediction}, there
remains flexibility along various axes in the analysis, namely:

\begin{enumerate}
\item the number of involved classes in the classification task,
\item the sorting and balancing of samples within the dataset,
\item the selected SVM hyperparameters, gamma and cost,
\item the uniformity (or possible lack of) SVM hyperparameters across binary classifiers,
\item the aggregation technique applied to binary classifiers and
\item the prediction method for the final labels.
\end{enumerate}
\TG{There is also flexibility in the evaluation metrics (macro vs micro)}

Considering the available degrees of freedom and limited computational resources, the AAC feature was used initially
to train and evaluate model parameters. The best performing model using AAC was then re-trained using the full feature
set. In the following section, the experimental design is described in detail with reference to the axes of flexibility,
above. Diagram~\ref{sec:supporingMaterials} provides a graphic representation of the parameters explored in this
section alongside the process through which the study was conducted.

\subsection{Experimental Design}
\label{sec:experimentaldesign}
Despite the multi-class nature of this task, the models developed and evaluated below were constructed in a binary
classification scenario. This was accomplished using the ``one versus rest'' strategy which was performed either prior
to training or automatically depending on the classifier. Support Vector Machine (SVM) Classifiers were initially built
using the SVMLight library, originally used by~\cite{mishra2014prediction}, which reported a probability of class
membership in each binary setting which were then combined as a multi-class confusion matrix. These models were
replicated using SciKit-learn (SKlearn), a popular library for machine learning in Python, in both an identical setting
to SVMLight (termed: SKlearn Probability) and an approach which automatically performs the class reconstruction and
prediction described above (termed: SKlearn Prediction).

\subsubsection{Training}
All models were fit for both $7$ and $8$ class scenarios (Addr: 1) -- excluding and including non-transport proteins,
respectively. The models were fit using $3$ distinct training paradigms: i) balanced, ii) shuffled, and iii)
downsampled (Addr: 2). In the balanced case, training- and testing-sets were created for each model through $5$-fold
cross validation (CV) that were randomly generated and stratified to balance class membership across folds. The
shuffled case was performed similarly to the balanced case without stratification guaranteeing balanced class membership
across folds. The downsampled case was also performed in accordance to the balanced case following a reduction in
samples to $60$ observations per class. This resulted in $6$ distinct training methods.

\subsubsection{Model Hyperparameters}
For all models, the Radial Basis Function (RBF) kernel was used and the gamma and cost parameters for the model ranged
from $1e^{-5}$ -- $10$ and $1$ -- $4$, respectively, consistent with those presented by~\cite{mishra2014prediction}.
Specific values were determined through a grid search (Addr: 3). In the case of SVMLight and SKLearn Probability
scenarios, gamma and cost values were either uniform or varied across classes (Addr: 4), whereas the implementation of
the SKlearn Prediction model permitted only uniform pairs across all classes.

\subsubsection{Performance Evaluation}
Model performance was evaluated through standard measures of sensitivity, specificity, accuracy, true positives (TP),
false positives (FP), true negatives (TN), false negatives (FN), and Matthew's Correlation Coefficient (MCC) which is a
measure of correlation suitable for imbalanced classification problems~\cite{mcc2017optimal}. As micro- and
macro-averaging approaches -- evaluating binary classifiers before or after aggregation into a multi-class model,
respectively -- lead to different results in an imbalanced classification setting, both approaches were used for
scoring here (Addr: 5).

In the case of both the SVMLight and SKlearn Probability models, the resulting classification and performance for each
model was determined by the aggregation of independent binary classifiers according to three distinct methods: maximum
probability, unweighted average, and balanced average (Addr: 6). The maximum probability method assigns each sample a
label corresponding to the binary classifier with the highest certainty, resulting in a non-overlapping classification
result which was evaluated. In the case of both averaging methods, each probability is converted into a  binary
classification through thresholding and is scored independently prior to aggregating the performance of all classifiers.
The unweighted average thresholds probabilities at the median value, whereas the balanced case uses a threshold
proportional to the true number of members belonging to a given class. As the SKlearn Prediction model returned
pre-determined group confusion matrix and class memberships, performance metrics were computed upon these directly.


\GK{Unrelated to here, but it would be good to replace or augment tables 2 and 3 with a contour density plot!}
