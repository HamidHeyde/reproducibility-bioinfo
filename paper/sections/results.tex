\section {Results}
\label{sec:results}

\subsection{Scikit-learn prediction-based models}
        
    Table \ref{tab:table1} shows the average accuracy, sensitivity, specificity and Mathews correlation coefficient values for different 
    scikit-learn prediction-based models being built on seven and eight substrate-specific classes for amino acid composition (AAC) 
    feature being averaged using both micro and macro averaging techniques.
    The reported metrics are obtained from running `SVC' classifier from scikit-learn library. We performed classification using both 
    `OneVsRestClassifier' and `SVC' functions as mentioned in \ref{subsub:scikitPredAlg}. The results were very close, so, we picked 
    `SVC' from `sklearn.svm' package as the classifier for our study for this section.\\

    The performance metrics with the least distance from the original ones is obtained by applying micro averaging techinqe to the 
    the results of the model with a shuffled eight class-based dataset. Among the models based on datasets with seven classes, 
    the shuffled one also provides the results with the least distance from the original one when final results are calculated 
    using micro averaging techinqe. On the same model mentioned above, the model with 8 classes shows better performance 
    compared to the 7 class-based sets. That is due to the fact that, there are more information available 
    (1380 vectors compared to 780) through the training phase. Oveall, micro and macro-averaging produce different results 
    for the MCC value. But it does not have a noticeable impact on the final results in case of down-sampled instance.\\

    Considering the imbalance nature of our data, when we divide the sorted instance of the main dataset into 5 different folds, 
    in each fold, there will be at least one class which is present in the test set but not available in the training set. That means, 
    the algorithm should predict a label it hasn’t seen before through the training phase.\\

    In the case of the shuffled instance of the main dataset, we can see that in the train and test sets of 
    each fold, there is a portion of each class present. So, the algorithm would be trained on all the 
    transporter and non-transporter classes. But the number of data points from each class available in either train or 
    test set would be different in each fold.\\

    In the case of the down-sampled instance of the main dataset, we have created a balanced set with an equal number of 
    sequences from each class. The problem is that we lose some information throughout the process. We would have 420 items 
    for 7 class-based sets (compared to 780 items before down-sampling) and 480 elements for 8 class-based sets 
    (compared to the 1380 elements before downsampling). So, the results are more balanced with lower values compared to the 
    shuffled one.\\

% Table 1
% 
\begin{table}[ht]
    \centering
    \begin{tabular}{|L | V |V V V V g g g V V V V | V |}
        \hline
        \multicolumn{14}{|g|}{Scikit-learn prediction-based models}\\
        \hline
        \multicolumn{1}{|l|}{\multirow{2}{*}{\footnotesize{Original Results}}}
        &
        \multicolumn{3}{g}{Accuracy} & \multicolumn{3}{g}{Sensitivity} &
        \multicolumn{3}{g}{Specificity} & \multicolumn{4}{g|}{MCC}\\
        \cline{2-14}&
        \multicolumn{3}{C}{73.74} & \multicolumn{3}{C}{74.65} &
        \multicolumn{3}{C}{73.22} & \multicolumn{4}{C|}{0.46} \\
        \hline
        
        \cline{1-12}\multicolumn{1}{|g}{}&
        
        \multicolumn{1}{|g|}{\footnotesize{Dist}}&
        \multicolumn{4}{g} {\footnotesize{7 class-based models}}&
        \multicolumn{3}{g}{}&
        \multicolumn{4}{g|} {\footnotesize{8 class-based models}}&
        \multicolumn{1}{g|}{\footnotesize{Dist}}\\
        
        \cline{3-6}\cline{10-13}\multicolumn{1}{|g}{}&
        \multicolumn{1}{|g|}{\footnotesize{ance}}&
        \multicolumn{1}{g}{acc}&\multicolumn{1}{g}{sens}&
        \multicolumn{1}{g}{spec}&\multicolumn{1}{g}{mcc}&
        \multicolumn{3}{g}{}&
        \multicolumn{1}{g}{acc}&\multicolumn{1}{g}{sens}&
        \multicolumn{1}{g}{spec}&\multicolumn{1}{g|}{mcc}&\multicolumn{1}{g|}{\footnotesize{ance}}\\

        \hline

        \multicolumn{1}{|l|}{\multirow{6}{*}{\footnotesize{Prediction-based}}}
        
        & 0.70 & 76.88 & 19.10 & 86.51 & 0.05 &    s&&s                & 84.58 & 38.33 & 91.19 & 0.29 & 0.45 \\
        & 0.39 & 83.66 & 42.86 & 90.47 & 0.33 &    d&\small{Micro}&d   & 84.79 & 39.16 & 91.31 & 0.30 & 0.44 \\
        & \mrkm{0.36}{84.43}{45.51}{90.91}{0.36}&    sh&&sh              & \mrkm{88.71}{54.85}{93.54}{0.48}{0.32} \\

        
        \cline{2-5}\cline{7-9}\cline{11-14}
        
        & 0.81 & 76.88 & 6.69 & 81.70 & 0.01 &    s&&s                & 84.58 & 76.82 & 86.79 & 0.03 & 0.81 \\
        & 0.39 & 83.67 & 42.85 & 90.47 & 0.32 &    d&\small{Macro}&d   & 84.79 & 39.16 & 91.31 & 0.29 & 0.44 \\
        & 0.39 & 85.27 & 43.31 & 90.33 & 0.35 &    sh&&sh              & 88.71 & 35.32 & 91.94 & 0.29 & 0.48 \\
        
        \hline\hline
        
        \multicolumn{14}{|c|} {\footnotesize{
            S, D and SH are sorted, down-sampled and shuffled instances of the main dataset.
        }}\\
        \multicolumn{14}{|c|} {\footnotesize{
            Acc: Accuracy, Sens: Sensitivity, Spec: Specificity, Mcc: Matthews correlation coefficient
        }}\\

        \hline
        
       

    \end{tabular}
    \captionsetup{font=small,width=14cm}
    \caption{The average sensitivity, specificity, accuracy, and MCC values 
    for scikit-learn prediction-based models for amino acid composition (AAC).}
    \label{tab:table1}
    
\end{table}

\subsection{SVMLight, Scikit-learn probability-based models}
    Table \ref{tab:table4} and \ref{tab:table5} show the average accuracy, sensitivity, specificity and 
    Mathews correlation coefficient values for all the 144 models being built on seven and eight class-based datasets 
    through our probability-based approach using the SVMLight application and the `SVC' method from Scikit-learn library. 
    The resulted performance metrics from both libraries for the models being based on seven transporter class datasets 
    (780 sequences) are presented in Table \ref{tab:table4} while the results from models based on eight transporter and non-transporter 
    classes (1380 sequences) are presented in Table \ref{tab:table5}.\\

    Compared to the original results, some models produced higher values for accuracy and specificity, a lower value for 
    sensitivity and a MCC value close to the reported one. Some others produced close values for accuracy, sensitivity 
    and specificity but lower value for the MCC. So, we introduced a distance value as a measure for making a comparison 
    across all the 144 models. The distance value for each model in Table \ref{tab:table4} and \ref{tab:table5}, 
    is the distance of the results of that specific model from the original reported values.\\
    
    Overall, the 8 class-based models produced higher values for all 4 involved performance metrics compared to the same models 
    being built on the dataset with 7 transporter classes due to having 600 more sequences in the main dataset which means more 
    information through the training phase. But higher values also means a greater distance from the original results. 
    With 7 class models we managed to obtain close balanced values for accuracy, sensitivity and specificity, 
    but the MCC was always less than the original one. On the other hand, when we based our models on the 8 class-based set, 
    we achieved higher values for all those four metrics.
    So, adding the non-transporter class provides better results and a closer MCC value while increasing the distance 
    in between the original and the reproduced results.\\
    
    Regarding the different instances of a dataset (balanced, shuffled and down-sampled version of an i.e. seven class-based set), 
    the results for balanced and shuffled instances are very close. That is due to the fact that although the number of sequences 
    (form each class) available in the train and test sets could be different from a balanced set to a shuffled one, 
    but the amount of difference is not hight. Balanced and shuffled instances are both imbalanced sets and are created as 
    mentioned in  \ref{sub:svmmodels}.
    The down-sampled instance is the balanced version of main dataset containing less sequences (420 data points compared to 780 
    for the 7 class-based sets and 480 elements compared to 1380 for the 8 class-based sets). 
    So, due to the availability of less but balanced information, we obtained more balanced results 
    but the values for metrics are less than those of the shuffled and balanced ones.\\ 
    
    Considering the imbalanced nature of shuffled and balanced instances, micro and macro-averaging produce different results 
    for these two sets. But it does not have a noticeable impact on the final results in case of down-sampled instance. 
    Micro averaging provides higher values for sensitivity and MCC metrics which leads to more balanced result 
    with less distance from the original ones.\\
    
    Assigning different Gamma and Cost value pairs to each class, produces better results compared to assigning 
    one single value pair to all classes. That is due to the fact that assigning different values to each class that is 
    accustomed to the distribution of values in that specific class, produces more true positives for that class after 
    classification which leads to a higher number for true positives for that fold. 
    But overall the results are close in both cases. \\
    
    Compared to SVC function from scikit-learn, SVMLight performs better on class-based model. In threshold-based model, 
    Scikit-learn provides results closer to the original ones while SVMLight provides higher MCC value and they both produce 
    close results in vote-based model.\\
    
    Considering the imbalanced nature of the dataset, the class-based model could be used where 
    there could more than one right answer (label) to the problem (multi-label classification). 
    If that is the case, then in an imbalanced problem like ours, 
    applying a threshold to the classification results (threshold-based model) can balance out the calculated metrics 
    for the problem. The vote-based model is the case for multi-class classification problem where there could 
    be one right label for the sequence.\\
    
    Overall, the closest results are achieved with micro-averaging the results on “threshold-based” model with either 
    balanced or shuffled instance of the main dataset. Using seven transporters class dataset, the MCC value and the distance 
    between reproduced and original results are less than the ones of the same model being built on eight class-based dataset, 
    but the latter shows better performance and provides higher values for performance metrics.\\

% Table 1

\begin{table}[ht]
    \centering
    \begin{tabular}{|L | V |V V V V g g g V V V V | V |}
        \hline
        \multicolumn{14}{|g|}{Scikit-learn prediction-based models}\\
        \hline
        \multicolumn{1}{|l|}{\multirow{2}{*}{\footnotesize{Original Results}}}
        &
        \multicolumn{3}{g}{Accuracy} & \multicolumn{3}{g}{Sensitivity} &
        \multicolumn{3}{g}{Specificity} & \multicolumn{4}{g|}{MCC}\\
        \cline{2-14}&
        \multicolumn{3}{C}{73.74} & \multicolumn{3}{C}{74.65} &
        \multicolumn{3}{C}{73.22} & \multicolumn{4}{C|}{0.46} \\
        \hline
        
        \cline{1-12}\multicolumn{1}{|g}{}&
        
        \multicolumn{1}{|g|}{\footnotesize{Dist}}&
        \multicolumn{4}{g} {\footnotesize{7 class-based models}}&
        \multicolumn{3}{g}{}&
        \multicolumn{4}{g|} {\footnotesize{8 class-based models}}&
        \multicolumn{1}{g|}{\footnotesize{Dist}}\\
        
        \cline{3-6}\cline{10-13}\multicolumn{1}{|g}{}&
        \multicolumn{1}{|g|}{\footnotesize{ance}}&
        \multicolumn{1}{g}{acc}&\multicolumn{1}{g}{sens}&
        \multicolumn{1}{g}{spec}&\multicolumn{1}{g}{mcc}&
        \multicolumn{3}{g}{}&
        \multicolumn{1}{g}{acc}&\multicolumn{1}{g}{sens}&
        \multicolumn{1}{g}{spec}&\multicolumn{1}{g|}{mcc}&\multicolumn{1}{g|}{\footnotesize{ance}}\\

        \hline

        \multicolumn{1}{|l|}{\multirow{6}{*}{\footnotesize{Prediction-based}}}
        
        & 0.70 & 76.88 & 19.10 & 86.51 & 0.05 &    s&&s                & 84.58 & 38.33 & 91.19 & 0.29 & 0.45 \\
        & 0.39 & 83.66 & 42.86 & 90.47 & 0.33 &    d&\small{Micro}&d   & 84.79 & 39.16 & 91.31 & 0.30 & 0.44 \\
        & \mrkm{0.36}{84.43}{45.51}{90.91}{0.36}&    sh&&sh              & \mrkm{88.71}{54.85}{93.54}{0.48}{0.32} \\

        
        \cline{2-5}\cline{7-9}\cline{11-14}
        
        & 0.81 & 76.88 & 6.69 & 81.70 & 0.01 &    s&&s                & 84.58 & 76.82 & 86.79 & 0.03 & 0.81 \\
        & 0.39 & 83.67 & 42.85 & 90.47 & 0.32 &    d&\small{Macro}&d   & 84.79 & 39.16 & 91.31 & 0.29 & 0.44 \\
        & 0.39 & 85.27 & 43.31 & 90.33 & 0.35 &    sh&&sh              & 88.71 & 35.32 & 91.94 & 0.29 & 0.48 \\
        
        \hline\hline
        
        \multicolumn{14}{|c|} {\footnotesize{
            S, D and SH are sorted, down-sampled and shuffled instances of the main dataset.
        }}\\
        \multicolumn{14}{|c|} {\footnotesize{
            Acc: Accuracy, Sens: Sensitivity, Spec: Specificity, Mcc: Matthews correlation coefficient
        }}\\

        \hline
        
       

    \end{tabular}
    \captionsetup{font=small,width=14cm}
    \caption{The average sensitivity, specificity, accuracy, and MCC values 
    for scikit-learn prediction-based models for amino acid composition (AAC).}
    \label{tab:table1}
    
\end{table}

\begin{table}[ht]
    \centering
    \begin{tabular}{|L |V V V V g g g V V V V |}
        \hline
        \multicolumn{12}{|g|}{7-class based model for AAC}\\
        \hline
        \multicolumn{1}{|l|}{\multirow{2}{*}{\footnotesize{Original Results}}}
        &
        \multicolumn{2}{g}{Accuracy} & \multicolumn{3}{g}{Sensitivity} &
        \multicolumn{3}{g}{Specificity} & \multicolumn{3}{g|}{MCC}\\
        \cline{2-12}&
        \multicolumn{2}{C}{73.74} & \multicolumn{3}{C}{74.65} &
        \multicolumn{3}{C}{73.22} & \multicolumn{3}{C|}{0.46} \\
        \hline\hline
        
        \multicolumn{12}{|g|} {SVM Light}\\
        \hline\hline
        \cline{1-12}\multicolumn{1}{|g}{}&
        \multicolumn{4}{g} {Gamma-Cost different for each class}&
        \multicolumn{3}{g}{}&
        \multicolumn{4}{g|} {same Gamma-Cost for all classes}\\
        \cline{2-5}\cline{9-12}\multicolumn{1}{|g}{}&
        \multicolumn{1}{g}{accuracy}&\multicolumn{1}{g}{sensitivity}&
        \multicolumn{1}{g}{specificity}&\multicolumn{1}{g}{mcc}&
        \multicolumn{3}{g}{}&
        \multicolumn{1}{g}{accuracy}&\multicolumn{1}{g}{sensitivity}&
        \multicolumn{1}{g}{specificity}&\multicolumn{1}{g|}{mcc}\\
        \hline

        \multicolumn{1}{|l|}{\multirow{6}{*}{\footnotesize{Before Voting}}}
        
        & 82.28 & 56.53 & 86.58 & 0.37 &    n&&n                & 80.73 & 60.00 & 84.18 & 0.37 \\
        & 84.07 & 39.28 & 91.55 & 0.32 &    d&\small{Micro}&d   & 81.90 & 48.33 & 87.5 & 0.32 \\
        & 81.74 & 54.74 & 86.24 & 0.36 &    sh&&sh              & 80.42 & 59.61 & 83.88 & 0.36 \\
        
        \cline{3-4}\cline{6-8}\cline{10-11}
        
        & 82.28 & 43.76 & 84.31 & 0.31 &    n&&n                & 80.73 & 46.64 & 81.69 & 0.30 \\
        & 84.08 & 39.28 & 91.54 & 0.32 &    d&\small{Macro}&d   & 81.90 & 48.33 & 87.49 & 0.33 \\
        & 81.73 & 43.27 & 84.06 & 0.28 &    sh&&sh              & 80.42 & 46.96 & 81.49 & 0.29 \\
        
        \hline
        \multicolumn{1}{|l|}{\multirow{6}{*}{\footnotesize{Thresholding}}}

        & 74.04 & 74.74 & 73.93 & 0.36 &    n&&n                & 73.07 & 76.66 & 72.47 & 0.36 \\
        & 72.89 & 66.66 & 73.92 & 0.30 &    d&\small{Micro}&d   & 69.96 & 71.19 & 69.76 & 0.29 \\
        & 73.88 & 72.18 & 74.16 & 0.34 &    sh&&sh              & 72.91 & 75.38 & 72.50 & 0.35 \\
        
        \cline{3-4}\cline{6-8}\cline{10-11}

        & 74.04 & 63.58 & 70.80 & 0.28 &    n&&n                & 73.07 & 65.34 & 69.17 & 0.27 \\
        & 72.89 & 66.66 & 73.92 & 0.30 &    d&\small{Macro}&d   & 69.96 & 71.19 & 69.76 & 0.30 \\
        & 73.88 & 62.26 & 71.01 & 0.27 &    sh&&sh              & 72.91 & 64.46 & 69.24 & 0.26 \\
        
        \hline
        \multicolumn{1}{|l|}{\multirow{6}{*}{\footnotesize{After Voting}}}

        & 84.87 & 47.05 & 91.17 & 0.38 &    n&&n                & 84.54 & 45.89 & 90.98 & 0.36 \\
        & 84.35 & 45.24 & 90.87 & 0.36 &    d&\small{Micro}&d   & 83.87 & 43.57 & 90.59 & 0.34 \\
        & 84.32 & 45.12 & 90.85 & 0.35 &    sh&&sh              & 84.21 & 44.74 & 90.79 & 0.35 \\
        
        \cline{3-4}\cline{6-8}\cline{10-11}

        & 84.87 & 34.05 & 89.58 & 0.28 &    n&&n                & 84.54 & 33.59 & 89.48 & 0.27 \\
        & 84.35 & 45.23 & 90.87 & 0.34 &    d&\small{Macro}&d   & 83.87 & 43.57 & 90.59 & 0.33 \\
        & 84.32 & 33.52 & 89.14 & 0.27 &    sh&&sh              & 84.21 & 33.47 & 89.15 & 0.27 \\
        
        \hline\hline
        
        \multicolumn{12}{|g|} {Scikit Learn}\\
        \hline\hline

        \multicolumn{1}{|l|}{\multirow{6}{*}{\footnotesize{Before Voting}}}
        
        & 87.42 & 34.10 & 96.30 & 0.39 &    n&&n                & 80.25 & 30.90 & 88.48 & 0.19 \\
        & 87.07 & 24.28 & 97.53 & 0.33 &    d&\small{Micro}&d   & 86.62 & 25.00 & 96.90 & 0.31\\
        & 86.75 & 31.92 & 95.90 & 0.36 &    sh&&sh              & 87.01 & 30.89 & 96.36 & 0.36\\
        
        \cline{3-4}\cline{6-8}\cline{10-11}
        
        & 87.41 & 28.70 & 95.77 & 0.34 &    n&&n                 & 87.20 & 24.18 & 96.06 & 0.28 \\
        & 87.07 & 24.28 & 97.53 & 0.30 &    d&\small{Macro}&d   & 86.63 & 25.00 & 96.90 & 0.28 \\
        & 86.75 & 26.20 & 95.36 & 0.29 &    sh&&sh              & 87.014 & 24.19 & 95.84 & 0.27 \\
        
        \hline
        \multicolumn{1}{|l|}{\multirow{6}{*}{\footnotesize{Thresholding}}}

        & 74.52 & 35.26 & 81.07 & 0.14 &    n&&n                & 75.07 & 31.54 & 82.33 & 0.12 \\
        & 80.13 & 24.52 & 89.40 & 0.14 &    d&\small{Micro}&d   & 79.25 & 25.00 & 88.29 & 0.13\\
        & 73.07 & 32.81 & 79.78 & 0.10 &    sh&&sh               & 73.90 & 31.66 & 80.94 & 0.11\\
        
        \cline{3-4}\cline{6-8}\cline{10-11}

        & 74.52 & 30.74 & 81.54 & 0.13 &    n&&n                & 75.07 & 25.34 & 82.74 & 0.09\\
        & 80.13 & 24.52 & 89.40 & 0.17 &    d&\small{Macro}&d   & 79.24 & 25.00 & 88.29 & 0.15\\
        & 73.07 & 27.74 & 80.28 & 0.09 &    sh&&sh              & 73.90 & 25.36 & 81.40 & 0.08 \\
        
        \hline
        \multicolumn{1}{|l|}{\multirow{6}{*}{\footnotesize{After Voting}}}

        & 81.28 & 34.49 & 89.08 & 0.24 &    n&&n                & 80.25 & 30.90 & 88.49 & 0.19\\
        & 78.84 & 25.95 & 87.65 & 0.13 &    d&\small{Micro}&d   & 79.18 & 27.14 & 87.86 & 0.15\\
        & 80.66 & 32.308 & 88.72 & 0.21&    sh&&sh              & 80.51 & 31.79 & 88.63 & 0.20\\
        
        \cline{3-4}\cline{6-8}\cline{10-11}

        & 81.28 & 29.66 & 89.07 & 0.19 &    n&&n                & 80.26 & 24.75 & 88.53 & 0.13 \\
        & 78.84 & 25.95 & 87.65 & 0.14&     d&\small{Macro}&d   & 79.18 & 27.14 & 87.85 & 0.14\\
        & 80.65 & 27.29 & 88.71 & 0.15 &    sh&&sh              & 80.51 & 25.97 & 88.65 & 0.14 \\
        \hline
        
       

    \end{tabular}
    \captionsetup{font=small,width=12cm}
    \caption{The average sensitivity, specificity, accuracy, and MCC from all the 
    models being built on 7 class-based dataset. n, d and sh are normal, 
    downsampled and shuffled instances of the main dataset.}
    \label{tab:table4}
    
\end{table}
\begin{table}[ht]
    \centering
    \begin{tabular}{|L | V |V V V V g g g V V V V | V |}
        \hline
        \multicolumn{14}{|g|}{8-class based model for AAC}\\
        \hline
        \multicolumn{1}{|l|}{\multirow{2}{*}{\footnotesize{Original Results}}}
        &
        \multicolumn{3}{g}{Accuracy} & \multicolumn{3}{g}{Sensitivity} &
        \multicolumn{3}{g}{Specificity} & \multicolumn{4}{g|}{MCC}\\
        \cline{2-14}&
        \multicolumn{3}{C}{73.74} & \multicolumn{3}{C}{74.65} &
        \multicolumn{3}{C}{73.22} & \multicolumn{4}{C|}{0.46} \\
        \hline\hline
        
        \multicolumn{14}{|g|} {SVM Light}\\
        \hline\hline
        
        \cline{1-12}\multicolumn{1}{|g}{}&
        
        \multicolumn{1}{|g|}{\footnotesize{Dist}}&
        \multicolumn{4}{g} {Gamma-Cost different for each class}&
        \multicolumn{3}{g}{}&
        \multicolumn{4}{g|} {same Gamma-Cost for all classes}&
        \multicolumn{1}{g|}{\footnotesize{Dist}}\\
        
        \cline{3-6}\cline{10-13}\multicolumn{1}{|g}{}&
        \multicolumn{1}{|g|}{\footnotesize{ance}}&
        \multicolumn{1}{g}{acc}&\multicolumn{1}{g}{sens}&
        \multicolumn{1}{g}{spec}&\multicolumn{1}{g}{mcc}&
        \multicolumn{3}{g}{}&
        \multicolumn{1}{g}{acc}&\multicolumn{1}{g}{sens}&
        \multicolumn{1}{g}{spec}&\multicolumn{1}{g|}{mcc}&\multicolumn{1}{g|}{\footnotesize{ance}}\\

        \hline

        \multicolumn{1}{|l|}{\multirow{6}{*}{\footnotesize{Class-based}}}
        
        & 0.24 & 87.33 & 64.92 & 90.53 & \mrko{0.49} &    b&&b               & 86.22 & 63.40 & 89.48 & 0.46 & 0.23 \\
        & 0.49 & 85.86 & 34.16 & 93.24 & 0.30 &    d&\small{Micro}&d   & 85.02 & 40.83 & 91.34 & 0.32 & 0.42 \\
        & 0.24 & 87.02 & 63.98 & 90.31 & 0.48 &    s&&s                & 86.24 & 62.46 & 89.63 & 0.46 & 0.23 \\
        
        \cline{2-5}\cline{7-9}\cline{11-14}
        
        & 0.43 & 87.33 & 39.91 & 88.02 & 0.30 &    b&&b               & 86.22 & 34.78 & 86.71 & 0.25 & 0.48 \\
        & 0.49 & 85.85 & 34.16 & 93.24 & 0.28 &    d&\small{Macro}&d   & 85.02 & 40.83 & 91.34 & 0.30 & 0.42 \\
        & 0.44 & 87.02 & 38.81 & 87.71 & 0.28 &    s&&s                & 86.24 & 33.57 & 86.89 & 0.23 & 0.50 \\
        
        \hline
        \multicolumn{1}{|l|}{\multirow{6}{*}{\footnotesize{Threshold-based}}}

        & \mrkm{0.10}{79.98}{79.78}{80.01}{0.44} &    b&&b               & \mrkm{79.74}{77.97}{80.00}{0.43}{0.10} \\
        & 0.19 & 68.93 & 71.87 & 68.51 & 0.27 &     d&\small{Micro}&d   & 68.07 & 69.79 & 67.82 & 0.25 & 0.22 \\
        & \mrkm{0.09}{77.71}{79.93}{77.40}{0.41} &     s&&s                & \mrkm{78.31}{78.33}{78.31}{0.41}{0.09} \\
        
        \cline{2-5}\cline{7-9}\cline{11-14}

        & 0.21 & 79.59 & 57.34 & 76.24 & 0.28 &    b&&b               & 79.59 & 57.34 & 76.24 & 0.28 & 0.25 \\
        & 0.19 & 68.93 & 71.87 & 68.51 & 0.27 &     d&\small{Macro}&d   & 68.07 & 69.79 & 67.82 & 0.25 & 0.22 \\
        & 0.21 & 77.71 & 53.55 & 74.14 & 0.27 &     s&&s                & 78.31 & 58.42 & 74.72 & 0.26. & 0.26 \\
        
        \hline
        \multicolumn{1}{|l|}{\multirow{6}{*}{\footnotesize{Vote-based}}}

        & 0.31 & 88.91 & 55.65 & 93.66 & \mrko{0.49} &    b&&b               & 88.51 & 54.05 & 93.43 & 0.47 & 0.32 \\
        & 0.43 & 84.84 & 39.37 & 91.33 & 0.30 &    d&\small{Micro}&d   & 84.79 & 39.16 & 91.30 & 0.30 & 0.44 \\
        & 0.32 & 88.44 & 53.76 & 93.39 & 0.47 &    s&&s                & 88.33 & 53.33 & 93.33 & 0.46 & 0.32 \\
        
        \cline{2-5}\cline{7-9}\cline{11-14}

        & 0.54 & 88.91 & 29.87 & 91.65 & 0.26 &    b&&b               & 88.51 & 25.93 & 91.35 & 0.22 & 0.58 \\
        & 0.44 & 84.84 & 39.37 & 91.34 & 0.29 &    d&\small{Macro}&d   & 84.79 & 39.16 & 91.31 & 0.28 & 0.44 \\
        & 0.57 & 88.44 & 27.85 & 91.28 & 0.23 &    s&&s                & 88.33 & 24.93 & 91.22 & 0.20 & 0.60 \\
        
        \hline
        \hline
        
        \multicolumn{14}{|g|} {Scikit Learn}\\
        \hline
        \hline

        \multicolumn{1}{|l|}{\multirow{6}{*}{\footnotesize{Class-based}}}
        
        & 0.48 & 90.22 & 36.33 & 97.84 & 0.46  &    b&&b               & 90.11 & 35.22 & 97.92 & 0.45 & 0.49 \\
        & 0.72 & 88.10 & 12.18 & 98.86 & 0.23 &    d&\small{Micro}&d   & 87.75 & 9.63 & 98.86 & 0.19 & 0.76  \\
        & 0.48 & 90.10 & 36.52 & 97.68 & 0.45 &    s&&s                & 89.89 & 33.28 & 97.96 & 0.43 & 0.50 \\
        
        \cline{2-5}\cline{7-9}\cline{11-14}
        
        & 0.68 & 90.19 & 17.30 & 96.85 & 0.22 &    b&&b               & 90.10 & 15.26 & 96.98 & 0.17 & 0.71 \\
        & 0.74 & 88.10 & 12.21 & 98.86 & 0.18 &    d&\small{Macro}&d   & 87.75 & 9.71 & 8.86 & 0.13 & 0.78  \\
        & 0.68 & 90.07 & 17.17 & 96.68 & 0.20 &    s&&s                & 89.88 & 12.90 & 97.02 & 0.15 & 0.74 \\
        
        \hline
        \multicolumn{1}{|l|}{\multirow{6}{*}{\footnotesize{Threshold-based}}}

        & \mrkm{0.08}{77.08}{80.0}{76.66}{0.40}  &     b&&b               & \mrkm{77.70}{78.18}{77.63}{0.40}{0.08} \\
        & 0.22 & 67.91 & 69.16 & 67.73 & 0.25 &     d&\small{Micro}&d   & 68.75 & 67.28 & 68.96 & 0.25 & 0.23  \\
        & \mrkm{0.08}{76.66}{78.47}{76.40}{0.39} &     s&&s                & 74.36 & 79.41 & 73.64 & 0.37 & 0.09 \\
        
        \cline{2-5}\cline{7-9}\cline{11-14}

        & 0.24 & 77.08 & 59.13 & 73.10 & 0.27 &     b&&b               & 77.70 & 54.72 & 74.14 & 0.25 & 0.28 \\
        & 0.21 & 67.91 & 69.16 & 67.73 & 0.26 &     d&\small{Macro}&d   & 68.75 & 67.29 & 68.95 & 0.26 & 0.22  \\
        & 0.27 & 76.66 & 56.55 & 72.78 & 0.25 &     s&&s                & 74.36 & 56.27 & 0.07 & 0.22 & 0.30 \\
        
        \hline
        \multicolumn{1}{|l|}{\multirow{6}{*}{\footnotesize{Vote-based}}}

        & 0.31 & 89.33 & 57.31 & 93.90 & \mrko{0.51} &     b&&b               & 88.73 & 54.92 & 93.56 & 0.48 & 0.32 \\
        & 0.42 & 85.10 & 40.41 & 91.49 & 0.31 &     d&\small{Micro}&d   & 85.31 & 41.25 & 91.60 & 0.32 & 0.41  \\
        & 0.31 & 89.09 & 56.37 & 93.76 & 0.50 &     s&&s                & 88.15 & 52.60 & 93.22 & 0.45 & 0.33 \\
        
        \cline{2-5}\cline{7-9}\cline{11-14}

        & 0.47 & 89.32 & 36.12 & 91.94 & 0.34 &     b&&b               & 88.73 & 31.39 & 91.51 & 0.29 & 0.52 \\
        & 0.43 & 85.10 & 40.41 & 91.48 & 0.30 &     d&\small{Macro}&d   & 85.31 & 41.25 & 91.60 & 0.31 & 0.42  \\
        & 0.49 & 89.09 & 33.78 & 91.73 & 0.30 &     s&&s                & 88.15 & 28.05 & 91.14 & 0.23 & 0.56 \\
        \hline\hline
        
        \multicolumn{14}{|c|} {\footnotesize{
            B, D and S are balanced, down-sampled and shuffled instances of the main dataset.
        }}\\
        \multicolumn{14}{|c|} {\footnotesize{
            Acc: Accuracy, Sens: Sensitivity, Spec: Specificity, Mcc: Matthews correlation coefficient
        }}\\

        \hline
        
       

    \end{tabular}
    \captionsetup{font=small,width=14cm}
    \caption{The average sensitivity, specificity, accuracy, and MCC  for 8 class-based models.}
    \label{tab:table5}
    
\end{table}



\begin{table}[ht]
    \centering

    \begin{tabular}{|L || g g g g || C C C C|}
    \hline
    \multicolumn{1}{|l||}{\multirow{2}{*}{Method}}
    &
    \multicolumn{4}{g||}{Paper's Results}
    &
    \multicolumn{4}{c|}{Reproduced Results}
    \\ \cline{2-5} \cline{6-9}
    \multicolumn{1}{|c||}{}
    &
    Sensitivity  &  Specificity  &  Accuracy  &  MCC
    &
    Sensitivity  &  Specificity  &  Accuracy  &  MCC
    \\
    \hline \hline
    AAC & 74.65 & 73.22 & 73.74 & 0.46      &    79.98 & 79.78 & 80.01 & 0.44 \\
    DPC & 71.36 & 71.17 & 71.32 & 0.40 & -- & -- & -- & --  \\
    PHC & 70.63 & 70.42 & 70.60 & 0.38 & -- & -- & -- & --  \\
    AAI & 71.54 & 71.98 & 71.80 & 0.40 & -- & -- & -- & --  \\
    PSSM & 74.00 & 76.03 & 75.48 & 0.47 & -- & -- & -- & --  \\
    \hline
    AAC + AAI & 73.84 & 73.36 & 73.37 & 0.44 & -- & -- & -- & --  \\
    AAC + PHC & 73.56 & 73.39 & 73.40 & 0.44 & -- & -- & -- & --  \\
    AAC + DPC & 73.01 & 74.23 & 73.85 & 0.44 & -- & -- & -- & --  \\
    AAC + PSSM & 72.33 &  74.62 &  73.87 &  0.44 & -- & -- & -- & --  \\
    DPC + PHC & 72.15 &73.13 &72.90 &0.43 & -- & -- & -- & --  \\
    DPC + AAI &    70.28  & 72.09 &  71.67 &  0.40 & -- & -- & -- & --  \\
    DPC + PSSM &69.87 &72.06 &71.37 &0.40& -- & -- & -- & --  \\
    AAI + PHC & 69.32 &  71.50 &  70.79 &  0.38 & -- & -- & -- & --  \\
    AAI + PSSM & 76.19 &77.17 &76.69& 0.49 & -- & -- & -- & -- \\
    \hline
    AAC + DPC + AAI & 73.01 &  74.32  & 73.89&   0.44 & -- & -- & -- & --  \\
    AAC + DPC + PHC & 73.04& 75.36   &74.75  & 0.46& -- & -- & -- & --  \\
    AAC + DPC + PSSM & 73.43 &73.73 &73.64 &0.44& -- & -- & -- & --  \\
    AAC + AAI + PHC & 73.01  & 73.82 &73.51 &  0.44& -- & -- & -- & --  \\
    AAC + AAI + PSSM & 72.34 &  74.62& 73.90 &  0.44 & -- & -- & -- & --  \\
    \hline \hline
    \multicolumn{9}{|c|}{\multirow{3}{*}{\footnotesize AAC:amino acid composition, 
    DPC:dipeptide composition, PHC:physico-chemical class composition,}}\\
    \multicolumn{9}{|c|}{}\\
    \multicolumn{9}{|c|}{\footnotesize AAI:biochemical composition(AAIndex), PSSM:position specific scoring matrix}\\
    \hline
    \end{tabular}
    
    \captionsetup{font=small,width=12cm}
    \caption{The average sensitivity, specificity, accuracy, and MCC for all seven 
    substrate-specific transporter classes for different SVM models on main dataset 
    comparing original results with the results from our best reproduced model}
    \label{tab:table6}
\end{table}