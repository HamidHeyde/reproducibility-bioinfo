\section {Introduction}


Reproducibility, the ability to reproduce computational results using
identical data and software~\cite{peng2011reproducible}, is a cornerstone
of scientific methodology. In the past decade, however, several studies
revealed a widespread lack of results reproducibility, to the
point that the existence of a reproducibility crisis is now acknowledged in
various fields~\cite{baker2016reproducibility}.

Counter-measures were identified to improve results
reproducibility~\cite{sandve2013ten}, among which study
pre-registrations~\cite{chambers2015registered}, open data and software
sharing~\cite{wilkinson2016fair}, and best
practices~\cite{nichols2017best} to report computational experiments. The
latter, in particular, is critical given the important methodological
flexibility associated with computational
experiments~\cite{carp2012plurality}.

Machine Learning experiments are not immune to reproducibility
issues~\cite{raff2019step}, in particular given the flexibility available in
data pre-processing, train/test set definitions, algorithm selection and
parametrization, library implementations, and evaluation metrics. The resulting 
reproducibility challenges have implications in various
disciplines including bioinformatics, the primary focus of our study.

% \Hamid{Importance of membrane protein classification:}
{Membrane proteins are vital molecules that act as gatekeepers to a cell. 
It is estimated that one in every three proteins found in a cell is a membrane protein~\cite{cell2016Membranes}. 
In a living organism, they play several important roles such as: 
cell signaling,  transportation of  molecules and nutrients across the membrane of a cell, 
energy production and foreign bodies recognition~\cite{kozma2012pdbtm}. Considering the contribution of 
these molecules to cell functionalities, defects in membrane proteins could lead to 
different diseases~\cite{gromiha2014bioinformatics}. 
Today, almost half of the drugs target these proteins~\cite{butt2017treatise}. 
Due to the hydrophobic surfaces of these molecules and their lack of conformational stability, 
using conventional experimental methods for annotation of these proteins are time-consuming, 
costly and sometimes impossible. So, researchers have turned into computational 
intelligent techniques for annotation and prediction of the structure and functionalities of the membrane proteins
~\cite{gromiha2006discrimination,gromiha2008functional,ou2010classification,schaadt2012functional,butt2016prediction}. 
Year after year, with advances in technology, researchers can use cheaper and 
faster sequencing methods (more data for their problems), new computational intelligent 
techniques and software tools. In search for more accurate and generalizable results, 
reproducible studies allows applying the same technique on new datasets and new techniques 
on the initial ones.}

This work presents a reproducibility study of a classification problem with an imbalanced dataset 
involving multiple classes which is a common case when dealing with proteins in bioinformatics. 
We report our attempts to reproduce a membrane protein classification problem with an imbalanced 
dataset~\cite{mishra2014prediction}, showing the impact of methodological
flexibility on classification performance, and deriving best practices to
report Machine Learning results for similar problems. We explore
methodological options related to data preparation, hyperparameter tuning,
classifier implementation, aggregation of binary classifiers for
multi-class classification, and prediction method for final labels. The
resulting variations emulate the range of results that might be obtained by
reasonably skilled experimenters aiming at reproducing the same model.


\marginnote{Hamid, could you add a few sentences here to
explain why membrane protein classification is important?}The work in~\cite{mishra2014prediction}
 is a reference contribution that we selected given
the availability of its input data, the quality of its writing and methods
reporting, and its overall impact in the field. 
