\section {Introduction}


Reproducibility, the ability to reproduce a computational results using the
same data and processing tools as in the initial experiment~\cite{peng}, is
a cornerstone of scientific methodology. In the past decade, however,
several studies revealed a lack of results reproducibility in various
fields~\cite{xxx, narps}, to the point that the onset of a reproducibility crisis
was widely acknowledged. 

Several measures were identified to address this crisis, among which (1)
study pre-registrations~\cite{cortexguy}, to limit  p-hacking, (2) open
data and software sharing according to the FAIR (Findable, Accessible,
Interoperable, Reusable) principles~\cite{fair,stodden}, to reduce
technical and non-technical barriers to reproducibility studies, (3)
stricter reporting of computational experiments, including through Jupyter
notebooks~\cite{xxx}, checklists~\cite{cobidas, pineau} or structured
provenance formats~\cite{xxx}, to provide a comprehensive documentation of
experiment conditions. The latter, in particular, is critical given the
important methodological flexibility associated with computational
experiments, exemplified in studies such as~\cite{carp} where the number of
valid pipelines to process a dataset was found to be in the thousands.

Machine Learning experiments are not immune to reproducibility
issues~\cite{overview}, in particular given the flexibility available in
data pre-processing, train/test set definitions, algorithm selection and
parametrization, and evaluation metrics. 

[link with bioinfo]

This paper presents a reproducibility study of transmembrane protein
classification using Machine Learning techniques. We report on our attempts
to reproduce a classical paper of the field~\cite{ref}, showing the impact
of methodological flexibility on classification performance, and
highlighting associated best practices to report Machine Learning results
in this field.  

[Choice of the paper]

\tbm{Plos One: Ten Simple Rules for Reproducible Computational Research}

\tbm{We picked a paper with Imbalanced dataset problem}
