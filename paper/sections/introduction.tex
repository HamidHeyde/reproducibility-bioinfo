\section {Introduction}

\TG{Add a sentence starting with ``The goal of this paper is \ldots''}
\tbm{FAIR: Finable, Accessible, Interoperable, Reusable}
\tbm{Joelle Pineau:  Machine Learning Reproducibility Checklist}
\tbm{Plos One: Ten Simple Rules for Reproducible Computational Research}

\tbm{We picked a paper with Imbalanced dataset problem}
Reproducible results are an essential requirement for computational studies including those based on machine learning techniques. 
Researchers' studies are often based on previously published experiments being conducted by their team or other scientists 
through the same field of study. 
Throughout the process, they may apply a new computational technique or a new model to the problem, 
approach the same problem from a different perspective or add up their own contribution to the field. \\

Failing to achieve the based results after reproducing a reported experiment, can cause significant financial costs 
and loss of time depending on the conducted study. The problem can be often addressed either through running the same 
experiment in an environment different than the original one (caused by changes in technology and tools available to a 
researcher at the time the study was conducted) or failing to report some details which may not look important but 
can significantly affect the reproduced results. \\

The solution for this common problem is sharing the work's dataset, features, source code and dependencies with other 
researchers using available means. Considering the fact that this may not be possible all the time, we need to make 
sure to report on all the details involved in the process. Through this work, we conducted a reproducibility study on a paper 
in bio-informatics field to address the missing details that could affect the reproduced results and create a guideline for 
having a reproducible experiment.\\

The code base for this paper is available at the paper's \git{}{GitHub} repository.
