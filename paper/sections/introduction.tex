\section {Introduction}


Reproducibility, the ability to reproduce computational results using
identical data and software~\cite{peng2011reproducible}, is a cornerstone
of scientific methodology. In the past decade, however, several studies
revealed a widespread lack of results reproducibility, to the
point that the existence of a reproducibility crisis is now acknowledged in
various fields~\cite{baker2016reproducibility}.

Counter-measures were identified to improve results
reproducibility~\cite{sandve2013ten}, among which study
pre-registrations~\cite{chambers2015registered}, open data and software
sharing~\cite{wilkinson2016fair}, and best
practices~\cite{nichols2017best} to report computational experiments. The
latter, in particular, is critical given the important methodological
flexibility associated with computational
experiments~\cite{carp2012plurality}.

Machine Learning experiments are not immune to reproducibility
issues~\cite{raff2019step}, in particular given the flexibility available in
data pre-processing, train/test set definitions, algorithm selection and
parametrization, library implementations, and evaluation metrics. The resulting 
reproducibility challenges have implications in various
disciplines including bioinformatics, the primary focus of our study.

This paper presents a reproducibility study of membrane protein
classification. We report our attempts to reproduce the work
in~\cite{mishra2014prediction}, showing the impact of methodological
flexibility on classification performance, and deriving best practices to
report Machine Learning results for similar problems. We explore
methodological options related to data preparation, hyperparameter tuning,
classifier implementation, aggregation of binary classifiers for
multi-class classification, and prediction method for final labels. The
resulting variations emulate the range of results that might be obtained by
reasonably skilled experimenters aiming at reproducing the same model.

Membrane protein classification is an important problem in
bioinformatics\marginnote{Hamid, could you add a few sentences here to
explain why membrane protein classification is important?}. The work in~\cite{mishra2014prediction}
 is a reference contribution that we selected given
the availability of its input data, the quality of its writing and methods
reporting, and its overall impact in the field. 
